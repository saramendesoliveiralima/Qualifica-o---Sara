%%%%%%%%%%%%%%%%%%%%%
% Resumo em Portugues
%%%%%%%%%%%%%%%%%%%%%

\resumo


No desenvolvimento de software a garantia da qualidade é um fator de grande importância. O mercado de dispositivos móveis tem crescido nos últimos anos, bem como, a necessidade de apps com qualidade e mais \textit{features} disponíveis. Assim, diversos estudos são propostos pela literatura para mitigar este problema. A atividade de teste, em especial, os testes de regressão, apresentam-se como uma estratégia viável para lidar com a complexidade, e com a constante evolução dos \ac{APPS}, visto que o objetivo destes testes é garantir que as mudanças realizadas não alterem o comportamento do sistema. Embora a literatura tenha dedicado esforços à produzir evidências sobre a criação de novas técnicas de teste de regressão para Android - o sistema operacional mais popular para os dispositivos móveis - os estudos existentes são bastante limitados no que diz respeito a demonstrar evidências empíricas sobre técnicas implementadas por ferramentas aplicáveis no desenvolvimento de \ac{APPS} Android. Neste contexto, este estudo tem como objetivo realizar uma avaliação empírica das técnicas de seleção de teste de regressão implementadas por ferramentas aplicadas a \ac{APPS} Android. Para tanto, a pesquisa foi organizada em quatro etapas: realização de uma revisão estruturada da literatura sobre técnicas de teste de regressão para \ac{APPS} Android; aplicação de um \textit{Survey} e realização de entrevistas - para compreensão de como estudantes/profissionais realizam teste de regressão; e confecção de um \textit{Evidences Briefings} para reportar a análise dos dados do \textit{Survey} e das entrevistas, para profissionais e pesquisadores da área. Ao final da pesquisa, espera-se as seguintes contribuições: corpo de conhecimento sobre o uso de técnicas de teste de regressão para o desenvolvimento de \ac {APPS} Android; prover evidências empíricas sobre quais as técnicas de teste de regressão implementadas por ferramentas são mais adequadas para projetos de \ac{APPS} Android; identificar como a indústria realiza o teste de regressão em \ac{APPS} Android; e apresentar ferramentas de teste de regressão disponíveis para projetos Android.



% Palavras-chave do resumo em Portugues
\begin{keywords}
Evolução de Software; Qualidade de Software; Teste de Regressão; APPS Android; Ferramentas.
\end{keywords}

%%%%%%%%%%%%%%%%%%%
% Resumo em Ingles
%%%%%%%%%%%%%%%%%%%



\abstract


%\textit{
In software development, quality assurance is a factor of great importance. The mobile device market has grown in recent years, as well as the need for quality apps and more \textit {features} available. Thus, several studies are proposed in the literature to mitigate this problem. The test activity, in particular, the regression tests, presents itself as a viable strategy to deal with the complexity, and with the constant evolution of the \ac {APPS}, since the objective of these tests is to guarantee that the changes made do not change the behavior of the system. Although the literature has devoted efforts to producing evidence on the creation of new regression testing techniques for Android - the most popular operating system for mobile devices - existing studies are quite limited in terms of demonstrating empirical evidence about techniques implemented by tools applicable in the development of \ac {APPS} Android. In this context, this study aims to carry out an empirical evaluation of the regression test selection techniques implemented by tools applied to Android \ac {APPS}. To this end, the research was organized in four stages: conducting a structured review of the literature on regression testing techniques for Android \ac {APPS}; applying a \textit {Survey} and conducting interviews - to understand how students/professionals perform regression testing; and making a \textit {Evidences Briefings} to report the analysis of the data from the \textit {Survey} and the interviews, to professionals and researchers in the field. At the end of the research, the following contributions are expected: the body of knowledge on the use of regression testing techniques for the development of Android \ac {APPS}; provide empirical evidence on which regression testing techniques implemented by tools are most suitable for Android \ac {APPS} projects; identify how the industry performs regression testing on \ac {APPS} Android; and present regression testing tools available for Android projects.


% Palavras-chave do resumo em Ingles
\begin{keywords}
%\textit{
Software Evolution; Software quality; Regression Testing; APPS Android; Tools.%}.
\end{keywords}