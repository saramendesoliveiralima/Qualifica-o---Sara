%%%%%%%%%%%%%%%%%%%%%
% Resumo em Portugues
%%%%%%%%%%%%%%%%%%%%%

\resumo


No desenvolvimento de software, a garantia da qualidade é um fator de grande importância. O teste de software é uma das estratégias que pode ser utilizada. Com o processo constante de evolução de software, orientado pelas manutenções realizadas, para garantir que as mudanças realizadas não alterem o comportamento do sistema faz-se uso do Teste de Regressão. Testes de regressão apresentam-se como uma estratégia viável para lidar com a complexidade, e com a constante evolução, dos sistemas de software, incluindo os \ac{APPS}, que tem sido cada vez mais utilizados no cotidiano. Embora a literatura tenha dedicado esforços à produzir evidências sobre a criação de novas técnicas de teste de regressão para Android,  o sistema operacional mais popular para os dispositivos móveis, os estudos existentes são bastante limitados no que diz respeito a demonstrar evidências empíricas que garantam quais são as técnicas implementadas por ferramentas mais aplicáveis no desenvolvimento de \ac{APPS} Android. Neste contexto, este estudo tem como objetivo realizar uma avaliação empírica das técnicas de seleção de teste de regressão implementadas por ferramentas aplicadas a \ac{APPS} Android. Para tanto, a pesquisa foi organizada em quatro etapas: etapa conceitual; realização de uma revisão estruturada da literatura sobre técnicas de teste de regressão para \ac{APPS} Android; aplicação de um \textit{Survey} e de entrevistas, para compreensão de como estudantes / profissionais realizam teste de regressão; e realização de um estudo experimental sobre técnicas de teste de regressão implementadas por ferramentas para \ac{APPS} Android. Ao final da pesquisa, espera-se as seguintes contribuições: corpo de conhecimento sobre o uso de técnicas de teste de regressão para o desenvolvimento de \ac {APPS} Android; prover evidências empíricas sobre quais as técnicas de teste de regressão implementadas por ferramentas são mais adequadas para projetos de \ac{APPS} Android; identificar como a indústria realiza o teste de regressão em \ac{APPS} Android; e apresentar ferramentas de teste de regressão disponíveis para projetos Android.



% Palavras-chave do resumo em Portugues
\begin{keywords}
Evolução de Software; Qualidade de Software; Teste de Regressão; APPS Android; Ferramentas.
\end{keywords}

%%%%%%%%%%%%%%%%%%%
% Resumo em Ingles
%%%%%%%%%%%%%%%%%%%



\abstract


\textit{In software development, quality assurance is a major factor. Software testing is one of the strategies that can be used. With the constant process of software evolution, guided by the maintenance performed, to ensure that the changes made do not alter the system behavior, the Regression Test is used. Regression testing is a viable strategy for dealing with the complexity and constantly evolving software systems, including APPS, which has been increasingly used in everyday life. While the literature has devoted efforts to produce evidence on the creation of new regression testing techniques for Android, the most popular mobile operating system, existing studies are quite limited in demonstrating empirical evidence to ensure which ones are. the techniques implemented by most applicable tools in Android \ac {APPS} development. In this context, this study aims to perform an empirical evaluation of regression testing techniques implemented by Android \ac {APPS} tools. To this end, the research was organized in four stages: conceptual stage; conducting a systematic mapping of regression testing techniques for \ac {APPS} Android; applying a Survey and interviews, to understand how students / professionals perform regression testing; and conducting an experimental study on regression testing techniques implemented by tools for \ac {APPS} Android. At the end of the research, the following contributions are expected: body of knowledge on the use of regression testing techniques for the development of Android \ac {APPS}; provide empirical evidence on which tool-implemented regression testing techniques are best suited for Android \ac {APPS} projects; identify how industry conducts regression testing on \ac {APPS} Android; and present regression testing tools available for Android projects}.

% Palavras-chave do resumo em Ingles
\begin{keywords}
\textit{Software Evolution; Software quality; Regression Testing; APPS Android; Tools}.
\end{keywords}