%%%%%%%%%%%%%%%%%%%%%
% Resumo em Portugues
%%%%%%%%%%%%%%%%%%%%%

\resumo


No desenvolvimento de software a garantia da qualidade é um fator de grande importância. O mercado de dispositivos móveis tem crescido nos últimos anos, assim como a necessidade de aplicativos (ou simplesmente apps) com qualidade e mais \textit{features} disponíveis. Nos últimos anos, percebe-se uma crescente quantidade de estudos que apresentam soluções para problemas inerentes à demanda supracitada. Em se tratando de garantia de qualidade, a atividade de testes de software tem um papel de grande importância para o desenvolvimento de apps. Em particular, os testes de regressão apresentam-se como uma estratégia viável para lidar com a complexidade, e com a constante evolução dos apps, visto que o seu principal objetivo é garantir que as mudanças realizadas entre versões não alteram o comportamento do sistema. Embora a literatura tenha dedicado esforços para o desenvolvimento de novas técnicas de teste de regressão para Android - o sistema operacional mais popular para os dispositivos móveis - os estudos existentes são bastante limitados no que diz respeito a demonstrar evidências empíricas sobre técnicas implementadas por ferramentas aplicáveis no desenvolvimento de \ac{APPS} Android. Neste contexto, a presente investigação tem como objetivo realizar uma avaliação empírica das técnicas de seleção de teste de regressão implementadas por ferramentas aplicadas a \ac{APPS} Android. Para tanto, a pesquisa foi organizada em quatro etapas: realização de uma revisão estruturada da literatura sobre técnicas de teste de regressão para \ac{APPS} Android; aplicação de um \textit{Survey} e realização de entrevistas - para compreensão de como profissionais realizam teste de regressão; e confecção de um \textit{Evidences Briefings} para reportar a análise dos dados do \textit{Survey} e das entrevistas, para profissionais da área. Ao final da pesquisa, espera-se as seguintes contribuições: corpo de conhecimento sobre o uso de técnicas de teste de regressão para o desenvolvimento de \ac {APPS} Android; evidências empíricas sobre quais as técnicas de teste de regressão implementadas por ferramentas são mais adequadas para projetos de \ac{APPS} Android; evidências sobre como a indústria realiza o teste de regressão em \ac{APPS} Android; e uma análise crítica das ferramentas de teste de regressão disponíveis para projetos Android.



% Palavras-chave do resumo em Portugues
\begin{keywords}
Evolução de Software; Qualidade de Software; Teste de Regressão; APPS Android.
\end{keywords}

%%%%%%%%%%%%%%%%%%%
% Resumo em Ingles
%%%%%%%%%%%%%%%%%%%



\abstract


%\textit{
In software development, quality assurance is a factor of great importance. The mobile device market has grown in recent years, as needs quality apps (or only apps) and more features available. In recent years, there has been an increasing number of studies that present solutions to problems inherent to the aforementioned demand. When it comes to quality assurance, software testing activity plays a significant role in developing apps. In particular, regression testing presents itself as a viable strategy to deal with complexity and the constant evolution of apps since its main objective is to ensure that changes made between versions do not alter the system's behavior. Although the literature has devoted efforts to developing new regression testing techniques for Android - the most popular operating system for mobile devices - the existing studies are quite limited in terms of demonstrating empirical evidence about techniques implemented by tools applicable in the Android \ac {APPS} development. In this context, the present investigation aims to evaluate the regression test selection techniques implemented by tools applied to \ac {APPS} Android. To this end, the research was organized in four stages: conducting a structured review of the literature on regression testing techniques for Android \ac {APPS}; applying a \textit {Survey} and conducting interviews - to understand how professionals perform regression testing, and making a \textit {Evidences Briefings} to report the analysis of the data from the \textit {Survey} and the interviews, to professionals in the field. At the end of the research, the following contributions are expected: a body of knowledge on the use of regression testing techniques for the development of Android \ac {APPS}; empirical evidence on which regression testing techniques implemented by tools are most suitable for Android \ac {APPS} projects; evidence on how the industry performs regression testing on \ac {APPS} Android; and a critical analysis of the regression testing tools available for Android projects.


% Palavras-chave do resumo em Ingles
\begin{keywords}
%\textit{
Software Evolution; Software quality; Regression Testing; APPS Android.%}.
\end{keywords}