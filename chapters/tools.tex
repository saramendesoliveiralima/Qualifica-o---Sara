\xchapter{Ferramentas de Teste de Regressão para \ac{APPS} Android}{}

Neste capítulo, apresentamos as ferramentas de teste de regressão para \ac{APPS} Android, encontradas na literatura e no mercado, fazendo um comparativo entre elas.

\section{Android}

\section{Ferramentas}

Tanto a literatura, quanto o mercado, propõe ferramentas para automatizar o processo de testes em \ac{APPS} Android. As ferramentas podem ser gratuitas ou proprietárias, e estão focadas em diferentes tipos de testes no processo de desenvolvimento dos apps. A seguir elencamos algumas destas ferramentas.

\begin{enumerate}
    \item \textbf{APPIUM}
    


    \item \textbf{ATOM}
    \item \textbf{CALABASH}
    \item \textbf{CHATEM}
    \item \textbf{DETREDUCE}
    \item \textbf{ESPRESSO}
    \item \textbf{GUIDIFF}
    \item \textbf{KATALON}
    \item \textbf{KMAX}
    \item \textbf{KOBITON}
    \item \textbf{MONKEY}
    \item \textbf{MONKEY RUNNER}
    \item \textbf{RANOREX}
    \item \textbf{REDROID}
    \item \textbf{RETESTDROID}
    \item \textbf{ROBOLECTRIC}
    \item \textbf{ROBOTIUM}
    \item \textbf{SEE TEST}
    \item \textbf{SQUISH}
    \item \textbf{TELERIK}
    \item \textbf{TEMA}
    \item \textbf{TEST COMPLETE}
    \item \textbf{TESTINGBOT}
    \item \textbf{UFT}
    \item \textbf{UI AUTOMATOR}
    
\end{enumerate}

A Tabela \ref{table:androidtools} apresenta uma síntese das ferramentas elencadas.





\section{Síntese do Capítulo}