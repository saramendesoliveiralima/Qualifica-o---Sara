\xchapter{Estudo Experimental}{}
\acresetall 


Este capítulo apresenta a definição e o planejamento do experimento. O presente estudo experimental foi concebido e estruturado com base nos conceitos de engenharia de software experimental e na avaliação de métodos e ferramentas de engenharia de software fornecidos por \citeonline{Wohlin:2012:ESE:2349018}. A Seção \ref{sec:definicaodoexperimento} apresenta a definição do experimento; a Seção \ref{sec:planejamentodoexperimentope} aborda as etapas referentes ao planejamento do experimento; e a Seção \ref{sec:resultadosesperadospe} lista os resultados esperados.

\section{Definição do Experimento}\label{sec:definicaodoexperimento}

Esta seção apresenta o objetivo, questões de pesquisa e métricas a serem utilizadas neste estudo experimental. 

\begin{itemize}

\item \textbf{Objetivo}:
   Coletar evidências empíricas acerca da eficiência das técnicas de teste de regressão implementadas por ferramentas \textcolor{blue}{evidenciadas na literatura \cite{8453877}}.
   
    \item \textbf{Questões de Pesquisa}:
    
    \begin{enumerate}[label=\bf QP\arabic*,leftmargin=2.1cm]
       
        \item \textcolor{blue}{As técnicas de teste de regressão implementadas por ferramentas A (evidenciada na literatura) e B (evidenciada no \textit{survey})} são eficientes no que refere-se a capacidade de cobertura de código fonte após realização de uma mudança efetuada no software, quando aplicadas em um conjunto maior de \ac{APPS}, vide Tabela \ref{table:apps}?
        
        \item \textcolor{blue}{As técnicas de teste de regressão implementadas por ferramentas A (evidenciada na literatura) e B (evidenciada no \textit{survey})} são eficientes no que refere-se a detecção de casos de testes a serem reexecutados que estão relacionados a alguma mudança efetuada no software, quando aplicadas em um conjunto maior de \ac{APPS}, vide Tabela \ref{table:apps}?
        
        \item \textcolor{blue}{As técnicas de teste de regressão implementadas por ferramentas A (evidenciada na literatura) e B (evidenciada no \textit{survey})} são eficientes no que refere-se a redução do número de casos de testes a serem reexecutados após alguma mudança efetuada no software, quando aplicadas em um conjunto maior de \ac{APPS}, vide Tabela \ref{table:apps}? 
        
        \item \textcolor{blue}{As técnicas de teste de regressão implementadas por ferramentas A (evidenciada na literatura) e B (evidenciada no \textit{survey})} são eficientes no que refere-se a detecção de falhas após realização de alguma mudança efetuada no software, quando aplicadas em um conjunto maior de \ac{APPS}, vide Tabela \ref{table:apps}?
        
        \item \textcolor{blue}{As técnicas de teste de regressão implementadas por ferramentas A (evidenciada na literatura) e B (evidenciada no \textit{survey})} são eficientes no que refere-se a precisão (omissão de testes não reveladores de falhas) após realização de alguma mudança efetuada no software, quando aplicadas em um conjunto maior de \ac{APPS}, vide Tabela \ref{table:apps}?
        
    \end{enumerate}
    

    
    \item \textbf{Métricas}:
    
    As métricas utilizadas nesse estudo experimental são baseadas no \textit{framework} proposto por \citeonline{536955}:
    
    
    \begin{enumerate}[label=\bf M\arabic*,leftmargin=2.1cm]
       
        \item Cobertura de Teste (CT): Essa métrica fornece a fração de todos as \textit{features} (ou requisitos / casos de uso) cobertos por um número selecionado de casos de teste ou por um conjunto de testes completo após realização de alguma mudança efetuada no software. Onde \textbf{$C_o_v$} é, portanto, gerado por uma ferramenta de cobertura de código.
        
        \begin{center}
        \noindent\fbox
        { 
            \parbox{.8\textwidth}
            {
                \begin{center}
            
                \textbf{$CT = C_o_v$}
    
                \end{center}
            }
        }
        \end{center}
        
        \item Detecção de casos de teste (DCT): Essa métrica fornece a fração do conjunto de casos de teste que precisam ser reexecutados após uma modificação no software selecionados pela técnica e o conjunto de casos de teste que precisam ser reexecutados após uma modificação do software. Onde \textbf{$CTS$} é o total de casos de testes selecionados pela técnica, e \textbf{$TCT$} é o total de casos de testes que precisam ser reexecutados.  
        
        \begin{center}
        \noindent\fbox
        { 
            \parbox{.8\textwidth}
            {
                \begin{center}
            
                \textbf{$DCT (\%) = ({CTS}/{TCT})*100$}
    
                \end{center}
            }
        }
        \end{center}
        

        \item Redução de casos de teste (RCT): Essa métrica fornece a fração do conjunto de casos de teste a serem reexecutados selecionados pela técnica e o conjunto de todos os casos de teste a serem reexecutados. Onde \textbf{$CTR$} é o conjunto de casos de testes para reexecução selecionados pela técnica, e \textbf{$TTR$} é o conjunto de todos os casos de testes para reexecução.
        
        \begin{center}
        \noindent\fbox
        { 
            \parbox{.8\textwidth}
            {
                \begin{center}
            
                \textbf{$RCT (\%) = 100-(({CTR}/{TTR})*100$)}
    
                \end{center}
            }
        }
        \end{center}
        
        \item Detecção de Falhas (DF): Essa métrica fornece a fração entre o número de defeitos encontrados pela técnica após alguma mudança realizada no software e o número de defeitos reportados. Onde \textbf{$TDS$} corresponde ao total de defeitos encontrados pela técnica, e \textbf{$TDR$} corresponde ao total de defeitos reportados.
        
        \begin{center}
        \noindent\fbox
        { 
            \parbox{.8\textwidth}
            {
                \begin{center}
            
                \textbf{$DF (\%) = ({TDS}/{TDR})*100$}
    
                \end{center}
            }
        }
        \end{center}
        
        \item Omissão de testes não reveladores de falhas (OTNV): Essa métrica fornece a fração entre o número de testes não reveladores de falhas omitidos pela técnica e o número de testes não reveladores existentes. Onde \textbf{$TNRS$} corresponde ao número de testes não reveladores de falhas omitidos pela técnica, e \textbf{$TNRE$} corresponde ao número de testes não reveladores existentes.
        
        \begin{center}
        \noindent\fbox
        { 
            \parbox{.8\textwidth}
            {
                \begin{center}
            
                \textbf{$OTNV (\%) = ({TNRS}/{TNRE})*100$}
    
                \end{center}
            }
        }
        \end{center}
        
   
    \end{enumerate}

    


%{\color{red}LEIA ISSO \url{https://www.qasymphony.com/blog/64-test-metrics/}}

\end{itemize}   

\section{Planejamento}\label{sec:planejamentodoexperimentope}

A etapa de planejamento define a forma como o experimento será conduzido \cite{Wohlin:2012:ESE:2349018}.

\subsection{Seleção de Contexto}

O objeto de estudo deste experimento são as técnicas de teste de regressão para \ac{APPS} Android. O estudo será realizado utilizando códigos de sistemas \textit{open source} encontrados no repositório github.  O estudo consiste na aplicação das técnicas de teste de regressão implementadas para \ac{APPS} Android \textcolor{blue}{sendo a técnica A (evidenciada na literatura) e a técnica B (evidenciada no \textit{survey})} em um conjunto de \ac{APPS} Android escritos em Java conforme Tabela \ref{table:apps}. O estudo será executado pela própria pesquisadora, sem contar com participantes. Para realização deste estudo será montado um ambiente controlado, utilizando um equipamento com capacidade de software e hardware para executar as ferramentas necessárias. 


\subsection{Projeto Piloto}

Antes de realizar o estudo, será realizado um projeto piloto com a mesma estrutura definida neste planejamento. O projeto piloto será realizado em um projeto \textit{open source} de pequeno porte que se encontra no github. Com o projeto piloto espera-se identificar possíveis inconsistências e melhorias a serem implementadas no planejamento.

\subsection{Formulação de Hipóteses}

Em um experimento é necessário declarar o que pretende-se avaliar. Este estudo experimenta apresenta duas hipóteses:

\begin{itemize}

    \item \textbf{Hipótese Nula (H0)}:
    \textcolor{blue}{As técnicas de teste de regressão implementadas por ferramentas A (evidenciada na literatura) e B (evidenciada no \textit{survey})} não são eficientes quando aplicadas em um conjunto maior de \ac{APPS}, vide Tabela \ref{table:apps}.
    
    \item \textbf{Hipótese Alternativa (H1)}:
     \textcolor{blue}{As técnicas de teste de regressão implementadas por ferramentas A (evidenciada na literatura) e B (evidenciada no \textit{survey})} são eficientes quando aplicadas em um conjunto maior de \ac{APPS}, vide Tabela \ref{table:apps}.

\end{itemize}


\subsection{Variáveis}

As variáveis \textbf{independentes} deste estudo serão: as técnicas de teste de regressão implementas por ferramenta para \ac{APPS} Android, \textcolor{blue}{sendo a técnica A (evidenciada na literatura) e a técnica B (evidenciada no \textit{survey})}; e, os \ac{APPS} Android, vide Tabela \ref{table:apps}.


A variável \textbf{dependente} deste estudo será a \textbf{eficiência} das técnicas de teste de regressão implementadas por ferramenta para \ac{APPS} Android, \textcolor{blue}{sendo a técnica A (evidenciada na literatura) e a técnica B (evidenciada no \textit{survey})}.


\subsection{Seleção das Unidades Experimentais (Projetos)}

Para realização deste estudo serão utilizados um conjunto de \ac{APPS} Android open source escritos em Java conforme Tabela \ref{table:apps} \footnote{\url{https://medium.mybridge.co/38-amazing-android-open-source-apps-java-1a62b7034c40}} \footnote{A data de extração dos dados referentes aos \ac{APPS} foi dia 28/04/2019}. Utilizaremos o código fonte, incluindo os casos de testes, da versão original do sistema e das suas \textit{releases}.




\subsection{Design do experimento}

Após definição do problema e seleção das variáveis, podemos projetar o experimento. O tipo de design que será utilizado nesse estudo é o fator com dois tratamentos, onde queremos comparar os dois tratamentos entre si \cite{Wohlin:2012:ESE:2349018}. Nesse sentido, o fator são as técnicas de teste de regressão implementadas por ferramentas para \ac{APPS} Android, e os tratamentos serão as técnicas \textcolor{blue}{A (evidenciada na literatura) e a B (evidenciada no \textit{survey})}.


\subsection{Tarefas do experimento}

Para execução deste experimento será necessário a realização de algumas etapas:


\begin{enumerate}

    \item Projetar casos de teste para cada programa
    \item Gerar casos de teste para cada programa
    \item Aplicar ferramenta que mensura a cobertura de teste para cada programa
    \item Executar casos de teste para cada programa
    \item Injetar manualmente mutantes (ou seja, falhas de mutação) nessas versões naturais do programa (\textit{releases} disponíveis nos repositórios) para criar versões defeituosas
    \item Aplicar a técnica de teste de regressão
    \item Coletar dados referentes a: cobertura de teste; detecção de casos de teste; redução de casos de teste; detecção de falhas; e, omissão de testes não reveladores.

\end{enumerate}

\section{Resultados Esperados}\\\label{sec:resultadosesperadospe}

Ao final deste experimento, espera-se obter evidências empíricas sobre a eficiência das técnicas de teste de regressão implementadas por ferramentas para \ac{APPS} Android \textcolor{blue}{sendo a técnica A (evidenciada na literatura) e a técnica B (evidenciada no \textit{survey})} em um conjunto de projetos open source, vide Tabela \ref{table:apps}, no que refere-se a: cobertura de teste; detecção de casos de testes a serem reexecutados; redução do número de casos de testes a serem reexecutados; detecção de falhas; e, precisão (omissão de testes não reveladores de falhas).

\begin{landscape}

\begin{table}[h!]
    \centering
    \scriptsize
    \caption{Projetos candidatos}
    \label{table:apps}
    \def \arraystretch{1}
    \begin{tabular}{m{1cm}m{3cm}m{4cm}m{3cm}m{2cm}m{2cm}m{2cm}m{2cm}}
        \toprule
        \bf \# & \bf Projeto & \bf URL & \bf Domínio & \bf Commits & \bf Releases & \bf Versão Atual & \bf Data\\
        \midrule
        
        \text {1} & Android-oss & \url{https://bit.ly/2Wb6vC5} & Rede Social & 4,002 & 26 & v1.11.0 & 14/02/2019\\
        
        \text{2} & NewPipe & \url{https://bit.ly/2VumXjx} & Rede Social & 4,396 & 65 & v0.16.1 & 14/03/2019\\
        
        \text{3} & WordPress & \url{https://bit.ly/28ZFo6c} & Rede Social & 37,736 & 433 & v12.2 & 24/04/2019\\
        
        \text{4} & Plaid & \url{https://bit.ly/2XW7QgB} & Rede Social & 1,152 & 11 & v1.0.9 & 19/03/2019\\
        
        \text{5} & FastHub & \url{https://bit.ly/2J1szeD} & Rede Social & 1,792 & 59 & v4.6.7 & 30/03/2018\\
        
        \text{6} & Materialistic & \url{https://bit.ly/2VtAVSZ} & Rede Social & 1,728 & 67 & v3.3 & 26/03/2019\\
        
        \text{7} & Telecine & \url{https://bit.ly/2VE3KMJ} & Fotos e Vídeos & 263 & 17 & 1.6.2 & 21/06/2016\\
        
        \text{8} & LeafPic & \url{https://bit.ly/2LaOMtn} & Fotos e Vídeos & 1,355 & 10 & v0.6-beta-1 & 09/10/2016\\
        
        \text{9} & K-9 Mail & \url{https://bit.ly/2L9ziG6} & Produtividade & 8,432 & 367 & v5.600 & 02/09/2018\\
        
        \text{10} & The ownCloud & \url{https://bit.ly/1Kw8CC8} & Produtividade & 7,321 & 86 & v2.10.0 & 07/03/2019\\
        
        \text{11} & AmazeFileManager & \url{https://bit.ly/2DAqfYE} & Produtividade & 3,646 & 36 & v3.3.2 & 13/01/2019\\
        
        \text{12} & Omni-Notes & \url{https://bit.ly/2DCa7G6} & Produtividade & 2,874 & 119 & 6.0.0_Beta_7 & 20/01/2019\\
        
        \text{13} & Timber & \url{https://bit.ly/2VvUJ81} & Música & 587  & 4 & v1.6 & 12/12/2017\\
        
        \text{14} & Phonograph & \url{https://bit.ly/2UKrDgT} & Música & 1,533 & 20 & v1.3.0 & 16/04/2019\\
        
        \text{15} & Shuttle Music Player & \url{https://bit.ly/2UNM1O4} & Música & 1,089 & 103 & v2.0.11-beta2 & 01/04/2019\\
        
        \text{16} & Santa-tracker & \url{https://bit.ly/2GLkec6} & Jogos & 18 & 4 & jinglebells & 29/01/2019\\
        
        \text{17} & 2048 & \url{https://bit.ly/2J0p1cA} & Jogos & 88 & 16 & v2.2 & 21/01/2019\\
        
        \text{18} & Signal & \url{https://bit.ly/2Vv1hUI} & Mensagens & 4,134 & 435 & v4.38.2 & 16/04/2019\\
        
        \text{19} & Qksms & \url{https://bit.ly/2GCJFN6} & Mensagens & 1,895 & 82 & v3.6.4 & 16/04/2019\\
        
        \text{20} & ExoPlayer & \url{https://bit.ly/2k9SVNV} & Mídia & 5,589 & 136 & r2.9.6 & 21/02/2019\\
        
        \text{21} & VLC & \url{https://bit.ly/2XQiSDZ} & Mídia & 80,855 & 50 & 4.0.0-dev & 30/11/2017\\
        
        \text{22} & AntennaPod & \url{https://bit.ly/2DDtO05} & Mídia & 4,896 & 83 & v1.7.1 & 12/11/2018\\
        
        \text{23} & Bitcoin Wallet & \url{https://bit.ly/2GNj1lY} & Finanças & 3,377 & 292 & v7.08 & 26/04/2019\\
        
        \text{24} & GnuCash & \url{https://bit.ly/2IMgB9k} & Finanças & 1,728 & 113 & v2.4.0 & 15/06/2018 \\
        
        \text{25} & CleanArchitecture & \url{https://bit.ly/2fNkhVt} & Arquitetura & 244 & 6 & v0.9.5 & 26/12/2016\\

        \text{26} & Shadowsocks client & \url{https://bit.ly/2XQKQzu} & Utilitários & 2,965 & 148 & v4.7.4 & 08/04/2019\\
        
        \text{27} & SoundRecorder & \url{https://bit.ly/2GO0fuO} & Utilitários & 80 & 3 & v1.3.0 & 24/05/2017\\
        
        \text{28} & iosched & \url{https://bit.ly/2vs2nBO} & Referência & 2,256 & 7 & v2015 & 18/08/2017\\
        
        \text{29} & Wikipedia & \url{https://bit.ly/2Wf14C4}  & Referência & 9,007 & 193 & vlatest & 26/04/2019\\
        
        \text{30} & SeeWeather & \url{https://bit.ly/2ZPExOy} & Clima & 128 & 2 & V2.03 & 28/02/2016\\
        
        \bottomrule
    \end{tabular}
\end{table}

\end{landscape}