\xchapter{\textit{Survey}}{}
\acresetall 

\section{Apresentação}

Segundo \cite{PSK2001} um \textit{survey} é um sistema abrangente que tem como objetivo coletar informações para descrever, comparar ou explicar conhecimentos, atitudes e comportamentos. A literatura apresenta alguns que tem trazem a aplicação de teste de regressão na indústria \cite{Ali2019}, \cite{630875}.


O presente \textit{survey} tem como objetivo compreender como a indústria de desenvolvimento e testes de \ac{APPS} Android realiza teste de regressão. Neste estudo, será adotado como ferramenta de pesquisa, um questionário para coletar informações a cerca de como é realizado o teste de regressão no desenvolvimento de \AC{APPS} Android.

\subsection{Questões de Pesquisa}
Para alcançar o objetivo proposto por este estudo, foi definida à seguinte questão de pesquisa principal:
\leavevspace 

\begin{center}
    \noindent\fbox{ 
        \parbox{.8\textwidth}
        {
        \begin{center}
            
            \textbf{A indústria de desenvolvimento de \ac{APPS} Android realiza teste de regressão?}
        \end{center}
        }
    }
\end{center}

\vspace{.5em}

Esta questão foi refinada em quatro questões específicas:
\vspace{.5em}

\begin{enumerate}[label=\bf QP\arabic*,leftmargin=1.8cm]
    
    \item \textbf{A realização de testes é feita de forma manual ou automatizada?} Essa questão de pesquisa busca identificar se o processo de teste é feito de forma manual ou automatizada.
    
    \item \textbf{A atividade de testes é focada em testes funcionais ou estruturais?} Essa questão de pesquisa visa identificar qual é foco da atividade de testes, agrupando em testes funcionais e estruturais.
    
    \item \textbf{Ao atualizarem um \ac{APPS}, realizam testes?} Essa questão de pesquisa pretende identificar se a empresa realiza teste de regressão.
    
    \item \textbf{Utilizam alguma ferramenta para realização de teste de regressão?} Essa questão intenta identificar potenciais ferramentas para realização de teste de regressão.
    
    
\end{enumerate}



\section{Planejamento}

Para o planejamento desse \textit{survey} foi utilizado o design de observação proposto por \cite{Kitchenham:2002:PSR:566493.566495} utilizado para identificar o comportamento dos profissionais tendo como base eventos já ocorridos.


O público alvo deste \textit{survey} são profissionais que trabalham na indústria com testes de \ac{APPS} Android.


Para divulgar essa pesquisa far-se-á uso da seguinte metodologia:

\begin{enumerate}
    \item Elaboração do questionário a ser aplicado, vide Apêndice \ref{sec:formulariopesquisa};
    \item Criação de um formulário on-line com uso do pacote de aplicativo Google Docs, para disponibilizar o questionário;
    \item Validação do questionário a ser aplicado, por meio de realização de um projeto piloto;
    \item Divulgação do formulário on-line por meio de e-mail para empresas que trabalham com desenvolvimento / testes de \ac{APPS} Android;
    \item Coleta, análise e síntese dos dados, obtidos pelo preenchimento do formulário on-line.
\end{enumerate}


O questionário foi dividido em quatro partes:

\begin{enumerate}
    \item O primeiro refere-se ao termo de consentimento livre e esclarecido;
    \item O segundo destina-se a identificar o perfil dos respondentes, sendo composto por dez perguntas;
    \item O terceiro refere-se ao padrão de seleção e realização de teste de regressão;
    \item O quarto destina-se saber se o respondente tem interesse em receber os resultados e, se necessário, participar de outras etapas da pesquisa.
\end{enumerate}


\section{Resultados Esperados}

Com o desenvolvimento deste \textit{survey}, espera-se obter os seguintes resultados:

\begin{itemize}
    \item Identificar como a indústria de desenvolvimento de \ac{APPS} Android executa teste de regressão;
    \item Levantar ferramentas utilizadas pela indústria para teste de regressão em \ac{APPS} Android;
    \item Identificar se a indústria de \ac{APPS} Android utiliza alguma técnica para selecionar testes a serem reexecutados;
    \item Levantar o potencial, em termos de \textit{features} implementadas, das ferramentas de teste de regressão disponíveis para projetos Android, e levantar oportunidades para implementações futuras.
\end{itemize}

