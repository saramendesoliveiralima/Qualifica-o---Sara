\xchapter{\textit{Survey}}{}
\acresetall 

Após apresentar as técnicas de seleção de testes de regressão encontradas na literatura, e as ferramentas disponíveis para testes de \ac{APPS} Android, essas informações serão usadas neste \textit{survey} com o objetivo de compreender como testadores e desenvolvedores, realizam o teste de regressão em projetos de apps para Android, como forma de responder a QP3 da presente investigação. Neste capítulo encontram-se detalhes sobre o planejamento, execução e resultados obtidos.


\section{Apresentação e Questões de Pesquisa}\label{rqsurvey}

Segundo \cite{PSK2001} um \textit{survey} é um sistema abrangente que tem como objetivo coletar informações para descrever, comparar ou explicar conhecimentos, atitudes e comportamentos. %A literatura apresenta alguns trabalhos que tem trazem a aplicação de teste de regressão na indústria \cite{Ali2019}, \cite{8094467}, \cite{7102609}, \cite{630875}.%

Definimos as seguintes Questões de Pesquisa (RQ):

\begin{enumerate}[label=\bf QP\arabic*,leftmargin=1.8cm]
    
    \item \textbf{Quão automatizado é o teste de regressão de aplicativos Android na prática?} Esta questão de pesquisa tem como objetivo identificar se os testadores utilizam-se de ferramentas para testar aplicativos Android. Também deseja-se descobrir as ferramentas usadas.
    
    \item \textbf{Como é realizado o processo de teste durante e após a manutenção dos apps?} Esta questão de pesquisa tem como objetivo identificar se os testadores realizam teste de regressão, e com que frequência. Também procura-se entender se e quais estratégias de seleção de teste podem ser empregadas para reduzir a quantidade de testes reexecutados.
    
    \item \textbf{Quais são os motivos para não realizar testes de regressão após a manutenção de apps para Android?} Esta questão de pesquisa tem como objetivo entender os motivos pelos quais os testadores não realizam testes de regressão em sua prática.
    
\end{enumerate}

\section{Metodologia de Pesquisa}\label{rmsurvey}

Este \textit{Survey} tem como objetivo compreender como testadores e desenvolvedores, realizam o teste de regressão em projetos de apps para Android.

Esta seção abrange os detalhes do planejamento, procedimentos de execução e relatórios dos resultados desejados e alcançados. Utilizamos a metodologia proposta por Kasunic \cite{Kasunic2005DesigningAE} e aplicamos os princípios de pesquisa definidos por Kitchenham et al. \cite{Kitchenham:2002:PSR:566493.566495}.


\subsection{Identificação do Público Alvo}

Para obter resultados válidos, selecionamos como público-alvo estudantes e profissionais de diferentes estados do Brasil que trabalham com testes de aplicativos Android. Foram considerados três critérios:

\begin{enumerate}
    \item Pessoas que trabalham de forma autônoma; 
    \item Pessoas que trabalham / pesquisam em uma empresa; 
    \item Alunos pesquisando tópicos relacionados.
\end{enumerate}

\subsection{Design do Questionário}


O \textit{Survey} é dividido em quatro seções. A Seção I refere-se ao termo de consentimento livre e esclarecido. A seção II refere-se à identificação do participante. Consiste em 13 perguntas, a maioria das quais são obrigatórias (8 perguntas de múltipla escolha, 3 subjetivas e 2 de seleção). Essas perguntas se referem a informações demográficas, área de formação, conhecimento e experiência em testes de software e testes de aplicativos Android. A Seção III trata do processo de teste de aplicativos Android. Consiste em 16 perguntas, a maioria das quais são obrigatórias (7 perguntas de múltipla escolha, 3 perguntas subjetivas e 6 perguntas na caixa de seleção). Essas perguntas tratam de como o processo de teste de software é realizado pelos participantes, bem como o uso de ferramentas para automatizar o processo de teste e se eles estão ou não satisfeitos com as ferramentas existentes. Finalmente, a Seção IV refere-se às considerações finais. 87\% das perguntas eram obrigatórias. A seguir, descrevemos os objetivos de cada seção do questionário.


\begin{itemize}
    \item \textbf{Perfil dos entrevistados:} pretendemos identificar o perfil dos entrevistados, com informações sobre sexo, idade, escolaridade, área de formação, certificações de teste, nível de conhecimento e experiência em testes de software e, especificamente, em aplicativos Android.
    
    \item \textbf{Foco da atividade de teste:} procuramos identificar se os testes realizados são funcionais ou estruturais. Além disso, queremos identificar os tipos de testes realizados nos projetos atuais, como: teste de \ac{GUI}, teste de unidade, teste de integração, teste de regressão, etc.
    
    \item \textbf{Automação do processo de teste:} pretendemos identificar se os testes são realizados de forma manual ou automatizada. Além disso, queremos identificar o suporte ferramental utilizado pelos entrevistados. No questionário, listamos ferramentas para testar aplicativos Android, encontradas na literatura e aplicadas na indústria. Portanto, poderíamos investigar se as ferramentas listadas na literatura também são conhecidas e utilizadas pelos entrevistados. A Tabela \ref{table:androidtools} mostra o conjunto de ferramentas de teste para aplicativos Android sobre as quais perguntamos aos entrevistados.
    
    \item \textbf{Compreender o processo de teste de regressão:} pretendemos identificar se os nossos participantes realizam testes de regressão após realizar manutenção em aplicativos Android. Estamos interessados em descobrir se: (i) o processo é feito manualmente ou automatizado, (ii) é aplicada alguma técnica de seleção de caso de teste, (iii) são utilizadas ferramentas, encontradas na literatura ou aplicadas na indústria, (iv) se estão satisfeitos com as ferramentas existentes; e (v) se e por que consideram a execução de testes de regressão como relevante.
    
\end{itemize}

\subsection{Projeto Piloto}

Aplicamos um piloto com três participantes, dois profissionais e um pesquisador, todos trabalhando com testes de aplicativos Android. Para a versão piloto do questionário, adicionamos uma seção de avaliação em que os participantes puderam fazer contribuições sobre a estrutura do questionário e indicar sugestões de melhorias. A seção de avaliação apresentou questões relacionadas à adequação das questões em relação ao tema da pesquisa; o número de perguntas a serem respondidas; o tempo necessário para responder ao formulário; e a relevância da pesquisa, como segue:


\begin{enumerate}
    \item Em uma escala de 1 a 5, onde 1 é igual a ruim e 5 é igual a ótimo, como você avalia o número de perguntas no formulário?
    \item Em uma escala de 1 a 5, onde 1 é igual a ruim e 5 é igual a ótimo, como você avalia a adequação das perguntas do formulário ao tema "Processo de teste de aplicativos para Android"?
    \item Cite pontos para melhorias.
    \item Cite pontos positivos do questionário.
    \item Deixe um comentário sobre o questionário.
\end{enumerate}

O \textit{feedback} sugeriu pequenas modificações na estrutura do questionário. As mudanças incluíram: adicionar opções de resposta a várias perguntas, alterar palavras para melhorar o entendimento e destacar o objetivo da pergunta para diferenciar questões que têm contextos semelhantes. Com base nas melhorias identificadas, uma segunda versão do questionário foi gerada.


\subsection{Distribuição do Questionário}

Para criar o questionário, usamos a ferramenta online Formulários Google\footnote{\url{http://docs.google.com/forms}}. O processo de ampla divulgação desta pesquisa foi realizado em 20 de janeiro de 2020 e ficou disponível por 30 dias. O link do questionário foi enviado por e-mails individuais, listas de e-mail e mídias sociais. O questionário também foi enviado por e-mail para empresas que trabalham com o desenvolvimento de aplicativos Android. O formulário foi enviado juntamente com um texto de convite, com informações básicas sobre o objetivo do estudo e a importância do estudo. Os participantes também foram informados sobre as políticas de privacidade do estudo de maneira clara e detalhada.


\section{Resultados}\label{resultssurvey}

O \textit{Survey} coletou 32 respostas. Em relação à demografia dos entrevistados, 78,1\% eram homens e 21,9\% mulheres; 37,5\% estavam entre 20 e 25 anos, 28,1\% tinham entre 26 e 30 anos, 18,8\% tinham entre 31 e 35 anos, e 15,6\% tinham mais de 35 anos.

Em relação ao nível de escolaridade dos entrevistados, 18,8\% possuíam ensino médio; 3,1\% possuíam curso técnico; 15,6\% eram estudantes de graduação; 34,4\% possuíam diploma de ensino superior; 6,2\% tinham cursos de especialização; 18,8\%
eram Msc.; e 3,1\% eram doutores. Além disso, 87,5\% eram provenientes de Ciência da Computação ou áreas afins, 3,1\% de Engenharia Elétrica, 3,1\% de outras engenharia, e 6,3\% possuíam formação superior em outras áreas. A respeito de cursos e certificações em testes, 56,2\% possuem pelo menos um deles. A maioria dos cursos citados foram nas seguintes ferramentas: Angular, Apium, JUnit, Selenium, e SonarQube. A certificação citada foi a \textit{Certified Tester Foundation} (CTFL) da \textit{International Software Testing Qualifications Board}.

Em relação ao seu papel em projetos Android, 18,8\% dos entrevistados declararam que trabalham como autônomos; 62,5\% trabalha / pesquisa ou já trabalhou em uma empresa de software, 40,6\% eram estudantes da área e 3,1\% afirmaram que
eles nunca trabalharam em projetos de teste para aplicativos Android. Os entrevistados foram de sete estados diferentes no Brasil.

Em relação à experiência de trabalho em teste de software, 25\% dos entrevistados afirmaram ter até 1 ano de experiência, 28,1\% tinham entre 1 e 3 anos experiência, 28,1\% possuíam entre 3 e 5 anos de experiência, 15,6\% entre 5 e 10 anos e apenas 3,2\% tinham mais de 10 anos de experiência. Considerando sua experiência com testes de aplicativos Android, 56,3\% tinham até 1 ano de experiência, 21,9\% tinham entre 1 e 3 anos de experiência, 15,6\% entre 3 e 5 anos de experiência e 6,2\% entre 5 e 10 anos de experiência.

Quando perguntados sobre seu nível de conhecimento em testes de aplicativos Android, comparados para outros profissionais, 15,6\% responderam que consideravam seu nível de
muito baixo, 25\% baixo, 21,9\% regular, 21,9\% bom e 15,6\% excelente.

Além disso, em relação à principal função do projeto atual na empresa em que trabalham, foram mencionados: analista de qualidade, analista de sistemas, analista de teste ou engenheiro de teste, design e arquitetura de teste automatizado, coordenador de inspeção, coordenador de inspeção, desenvolvedor e gerente de projeto.

Em relação às atividades que mais realizam, 75\% trabalham com criação e design de casos de teste, 84,4\% trabalham com a execução de casos de teste, 3,1\% trabalham com automação de teste e 3,1\% trabalham com o relatório de teste para o cliente. Para esta pergunta, os entrevistados podem verificar mais de uma opção.

\subsection{Questões de Pesquisa}

Em seguida, relatamos os resultados da pesquisa, com base no conjunto das questões de pesquisa apresentadas anteriormente na na Seção \ref{rqsurvey}.

\begin{enumerate}[label=\bf QP\arabic*]
    
\item \textbf{Foco da atividade de teste:}
    
Observamos que a maioria dos testes realizados pelos entrevistados são testes funcionais, correspondendo a 90,6\%. Os testes estruturais são realizados por 25\% dos entrevistados. Os testes de “caixa cinza”, nos quais o acesso ao código fonte é parcial, são realizados por 21,9\% dos entrevistados.

Quando perguntado sobre os tipos de testes que eles executam nos projetos Android, o teste da GUI foi o mais executado e foi seguido pelos testes de validação, regressão, sistema, integração e unidade. A Figura \ref{figure:s_tipostestes} mostra a distribuição das categorias de teste. Vale ressaltar que na questão referida os respondentes puderam selecionar mais de uma opção.
    
\item \textbf{Automação do processo de teste:}

34,4\% dos entrevistados responderam que realizam testes do Android manualmente, sem usar nenhuma ferramenta de suporte; 15,6\% afirmaram realizar testes automatizados; e 50\% responderam que realizam testes manuais e automatizados. Além disso, além das ferramentas apresentadas no questionário (da literatura e da indústria), os entrevistados também podem indicar outros. A Figura \ref{figure:s_ferramentastestes} mostra que as ferramentas mais usadas são Appium, UI Automator, Monkey e Katalon. Além disso, alguns entrevistados relataram que é uma prática comum para suas empresas criar sua própria ferramenta de teste, para atender às suas demandas. Nesta pergunta, mais de uma opção pode ser verificada.
    
\item \textbf{Teste após manutenção (atualização) de apps Android:}

Quando perguntados sobre a execução de qualquer tipo de teste após as tarefas de manutenção, 40,6\% responderam que sempre realizam testes, 12,5\% responderam que realizam frequentemente, 21,9\% responderam que às vezes realizam testes, 9,4\% responderam que raramente testam e 15,6\% responderam que nunca testam.

Em relação à automação do processo de teste durante a manutenção, 25\% responderam que os testes são manuais, 15,6\% responderam que os testes são realizados de forma automatizada, 53,1\% responderam que realizam testes manualmente e automatizados e 6,3\% responderam que não testam os aplicativos após realizar a manutenção.

A literatura sobre técnicas de seleção para testes de regressão aponta alguns procedimentos que visam reduzir o número de testes a serem reexecutados \cite{KAZMI2017}, \cite{ENGSTROM201014}, \cite{Yoo:2012:RTM:2284811.2284813}, \cite{Rothermel2000}, \cite{536955}, \cite{Graves:2001:ESR:367008.367020}, \cite{WHITE1991}, \cite{65194}. Nesse sentido, pedimos aos participantes que entendessem quais desses procedimentos são realizados durante o processo de teste após o lançamento de uma nova versão do aplicativo.

Observamos que o processo mais executado é executar novamente todos os casos de teste da versão original para testar a versão atualizada do aplicativo. Essa é a técnica essencial do processo de teste de regressão, embora não seja a mais eficiente, principalmente se o número de casos de teste aumentar \cite{RothermelGreggHarroldMary2000}. A Figura \ref{figure:s_processostestemanutencao} mostra esses resultados.


Quanto ao uso de ferramentas no processo de teste de regressão, 31,2\% responderam que não usam ferramentas, enquanto 68,8\% responderam que utilizam ferramentas no processo de teste de regressão. Além disso, indicamos no questionário um conjunto de ferramentas disponíveis para testes de regressão de aplicativos Android e solicitamos aos participantes que indiquem quais eles costumam usar. Eles estavam livres para mencionar outras ferramentas, que não estão listadas no questionário. A Figura \ref{figure:s_ferramentastestenovo} apresenta o principal resultado para esta pergunta. Observamos que ATOM e Redroid são as ferramentas mais utilizadas (na literatura) e MonkeyRunner (na indústria).

Em relação à opinião dos participantes sobre a realização de testes após a manutenção, 96,9\% responderam que consideram isso uma tarefa relevante. Também perguntamos quais são os principais fatores para suas empresas não executarem testes após a manutenção dos aplicativos. Além dos fatores indicados na pesquisa, os entrevistados apontaram outros fatores, como “cultura no processo de desenvolvimento, acreditando que a fixação de uma área não afetará outra” e “pressão do tempo do cliente para lançar o produto no mercado”. mercado para não perder o tempo de colocação no mercado ”.

A Figura \ref{figure:s_fatorestestemanutencao} mostra os principais fatores relatados pelos entrevistados. Observamos que o fator que mais influencia o teste de regressão por não ser realizado é o tempo reduzido para entregar uma versão atualizada do aplicativo ao cliente.

Quando perguntados se as ferramentas utilizadas atendem às necessidades de realizar testes de regressão, 9,4\% responderam que nunca atendem, 9,4\% responderam raramente, 12,5\% responderam às vezes, 18,8\% responderam muitas vezes, 12,5\% responderam que sempre atendem e 37,4\% responderam que não testam o aplicativo após a manutenção. Quando perguntados se estavam satisfeitos com os recursos oferecidos pelas ferramentas de teste de aplicativos Android existentes, 43,8\% responderam que sim e 56,2 \% responderam que não.

\end{enumerate}

\section{Discussões}\label{discussions}

Nesta seção, discutimos os resultados em quatro aspectos: perfil do participantes, foco da atividade de teste, automação do processo de teste, e entender o processo de teste de regressão.

\textbf{Perfil dos entrevistados:} A maioria dos participantes era do sexo masculino, com idade entre 20 e 30 anos, e 50\% deles possuíam diploma de ensino superior em ciência da computação ou áreas afins. Mais de 50\% dos participantes tinham certificação na área de testes e eles tinham entre 1 a 5 anos de experiência profissional na área de testes. No entanto, com relação à experiência profissional na área específica de teste de aplicativos Android, a maioria dos participantes tem até 1 ano de experiência profissional e eles trabalham ou pesquisam em uma empresa de software. Observamos também que houve uma distribuição próxima entre as pessoas que
consideram seu conhecimento em testes de aplicativos Android como bom ou excelente e pessoas que consideram seu conhecimento baixo ou muito baixo. Para as funções realizadas nas empresas, são diversificadas e as atividades desenvolvidas são principalmente criação, design e execução de casos de teste. Apesar de haver muitos profissionais treinados na área de teste de software, relacionados a testes Aplicativos Android, ainda há uma necessidade maior de treinamento, devido ser uma área em expansão.
    
\textbf{Foco da atividade de teste:} Observamos que testes de caixa preta são utilizados na maioria dos projetos, sendo apontada por mais de 91\% dos participantes. A respeito de
as categorias de testes realizados, o teste \ac{GUI}, teste de validação, teste do sistema e teste de regressão se destacam. O teste da \ac{GUI} é o mais citado pelos participantes. Esse resultado era esperado, dada a importância da interface do software e sua usabilidade para dispositivos móveis.
    
\textbf{Automação do processo de teste:} Observamos que o processo de automação de teste ainda não é realizado na maioria das empresas de software pesquisadas, pois apenas 16\% dos participantes indicaram que realizam testes automatizados. Quanto às ferramentas, as que provêm do mercado foram relatadas como muito mais utilizadas do que as encontradas na literatura, apesar dos estudos empíricos publicados que mostram a eficácia das ferramentas que a comunidade de pesquisa propôs ao longo dos anos. De fato, existe uma grande lacuna a ser preenchida, no sentido de que as ferramentas disponíveis na literatura devem ser aprimoradas para atender às necessidades do setor, assim como seus benefícios anunciados e prováveis ganhos (em termos de custo, tempo e escopo) enfrentados pela indústria também.
Embora mais de 50\% dos entrevistados tenham apontado que não estão satisfeitos com as ferramentas existentes para o teste de regressão, eles não indicaram o conjunto de requisitos que uma ferramenta de regressão deve incluir para atingir seus objetivos.
Perguntamos aos participantes quais recursos poderiam ser aprimorados ou implementados nas ferramentas que eles estão usando atualmente. Em seguida, resumimos as respostas mais relevantes:
\begin{itemize}
    \item "O teste da interface do usuário do Android ainda é difícil e demorado."
    \item “A criação e execução de suítes de teste precisam ser automatizadas para permitir uma verificação mais rápida da qualidade do aplicativo após as alterações.”
    \item “Relatórios de execução de teste podem ser emitidos em relação a uma versão anterior do software para revelar casos de teste que não foram cobertos na nova versão, mas na versão anterior.”
    \item “Testes de \ac{GUI} (testes de interface do usuário) no Android podem ter ferramentas que facilitam principalmente o teste de fluxos completos para testes de ponta a ponta.”
\end{itemize}

Tais declarações apresentam dificuldades encontradas pelos participantes no uso das ferramentas de teste de aplicativos Android, o que pode representar alguns dos motivos pelos quais eles não costumam usar as ferramentas. 

\item \textbf{Compreensão do processo de teste de regressão:} Embora mais de 90\% dos participantes considerassem a execução do teste após a execução de manutenção em um aplicativo Android sendo relevante, apenas 69\% deles responderam que realizam testes neste momento no ciclo de vida do software. Segundo eles, há uma enorme pressão para entregar a versão atualizada, e o teste não é considerado como prioridade em seu processo.
No entanto, apenas 53\% dos participantes indicaram que realizam testes de regressão, o que significa que, talvez, alguns participantes não conseguiram associar o conceito de teste após a manutenção ao teste de regressão. Além disso, quando eles realizam testes de regressão, geralmente estes não são automatizados.
A técnica mais utilizada para realizar testes de regressão está relacionada à reexecução de todos os casos de teste da versão anterior, embora essa técnica seja apontada pela literatura como uma estratégia inviável \cite{RothermelGreggHarroldMary2000}. Além disso, os participantes não relataram usar nenhum processo para selecionar os testes mais eficazes a serem re executados após as atualizações do software. Por exemplo, uma estratégia seria reutilizar casos de teste existentes e excluir casos obsoletos.
Além disso, o baixo nível de experiência no teste de aplicativos Android relatado pelos participantes pode influenciar diretamente a maneira como eles executam o processo de teste. A maioria deles tinha até 1 ano de experiência com testes em apps Android, o que pode ser um dos motivos pelos quais ignoraram as ferramentas da literatura. Outras razões podem estar associadas à indisponibilidade de ferramentas, ou seja, não é comum que um projeto de pesquisa seja mantido atualizado, e ferramentas interessantes só estão sujeitas a discussão em uma única pesquisa, sem divulgá-la à prática da indústria.


\section{Ameaças à Validade}\label{threatstovalidity}


\textbf{Validade externa:} embora nosso estudo relate resultados com base em uma amostra que considerou pesquisadores e profissionais de teste de software, os resultados podem não ser generalizados para outros cenários. No entanto, este estudo pode ser considerado como um passo inicial para fornecer mais evidências sobre a conscientização e a percepção dos profissionais sobre o uso do teste de regressão na manutenção de aplicativos Android. Fornecemos os procedimentos e os dados de pesquisa para permitir mais replicações.

\textbf{Validade de construção:} no questionário, evitamos usar o termo testes de regressão. Em vez disso, usamos implicitamente a definição do teste de regressão, como uma estratégia para evitar vieses e impedir que os participantes pesquisem o conceito, antes de responder ao questionário. No entanto, alguns participantes podem não ter relacionado esse conceito aos testes de regressão.

\textbf{Validade de conclusão:} Neste estudo, não realizamos nenhuma análise quantitativa, pois a amostra não era grande o suficiente. Em vez disso, realizamos uma análise qualitativa, \textit{insights} sobre os pontos importantes a serem considerados, principalmente para futuras investigações.


\section{Síntese do capítulo}

O presente capítulo apresentou o \textit{Survey} utilizado como instrumento de pesquisa para compreender qual a perspectiva de acadêmicos e profissionais sobre técnicas de seleção de teste de regressão implementadas por ferramentas para projetos de \ac{APPS} Android. Na Seção \ref{rqsurvey} definimos as questões de pesquisa. Na Seção \ref{rmsurvey} apresentamos a metodologia utilizada. A Seção \ref{resultssurvey} apresenta os resultados obtidos, e na Seção \ref{discussions} as devidas discussões. Por fim, na Seção \ref{threatstovalidit} apontamos possíveis ameaças à validade deste estudo.




























