\xchapter{\textit{Survey}}{}
\acresetall 

Após apresentar as técnicas de seleção de testes de regressão encontradas na literatura, e as ferramentas disponíveis para testes de \ac{APPS} Android, essas informações serão usadas neste \textit{survey} com o objetivo de compreender como o processo de testes de apps Android é realizado por estudante e profissionais que trabalham nessa área, bem como, verificar se as ferramentas encontradas tanto na literatura quanto no mercado são utilizadas por eles, como forma de responder a QP3 da presente investigação. Neste capítulo encontram-se detalhes sobre o planejamento, execução e resultados obtidos.


\section{Apresentação e Questões de Pesquisa}\label{rqsurvey}

Segundo \cite{PSK2001} um \textit{survey} é um sistema abrangente que tem como objetivo coletar informações para descrever, comparar ou explicar conhecimentos, atitudes e comportamentos. A literatura apresenta alguns trabalhos que tem trazem a aplicação de teste de regressão na indústria \cite{Ali2019}, \cite{8094467}, \cite{7102609}, \cite{630875}.

Definimos as seguintes Questões de Pesquisa (RQ):

\begin{enumerate}[label=\bf QP\arabic*,leftmargin=1.8cm]
    
    \item \textbf{Qual é o foco da atividade de teste realizada pelos testadores?} Esta questão de pesquisa tem como objetivo identificar se os testes realizados são funcionais e/ou estruturais, além de identificar quais tipos de testes são realizados.
    
    \item \textbf{Os testes são realizados manualmente ou automaticamente?} Esta questão de pesquisa tem como objetivo identificar a maneira como os testes são realizados, manualmente ou automatizados. Também queremos identificar quais ferramentas são usadas no processo de teste.
    
    \item \textbf{Ao executar a manutenção (atualização) em um aplicativo, os testadores realizam testes que garantem que o aplicativo não alterou seu comportamento?} Esta questão de pesquisa tem como objetivo identificar se eles realizam testes de regressão e como esse processo é executado.
    
\end{enumerate}

\section{Metodologia de Pesquisa}\label{rmsurvey}

Este \textit{Survey} tem como objetivo compreender o processo de teste de aplicativos Android, o conhecimento e o uso das ferramentas encontradas na literatura e aquelas amplamente aplicadas na indústria e o processo de teste de regressão. 

Esta seção abrange os detalhes do planejamento, procedimentos de execução e relatórios dos resultados desejados e alcançados. Utilizamos a metodologia proposta por Kasunic \cite{Kasunic2005DesigningAE} e aplicamos os princípios de pesquisa definidos por Kitchenham et al. \cite{Kitchenham:2002:PSR:566493.566495}.


\subsection{Identificação do Público Alvo}

Para obter resultados válidos, selecionamos como público-alvo estudantes e profissionais de diferentes estados do Brasil que trabalham com testes de aplicativos Android. Foram considerados três critérios:

\begin{enumerate}
    \item Pessoas que trabalham de forma autônoma; 
    \item Pessoas que trabalham / pesquisam em uma empresa; 
    \item Alunos pesquisando tópicos relacionados.
\end{enumerate}

\subsection{Design do Questionário}


O \textit{Survey} é dividido em quatro seções. A Seção I refere-se ao termo de consentimento livre e esclarecido. A seção II refere-se à identificação do participante. Consiste em 13 perguntas, a maioria das quais são obrigatórias (8 perguntas de múltipla escolha, 3 subjetivas e 2 de seleção). Essas perguntas se referem a informações demográficas, área de formação, conhecimento e experiência em testes de software e testes de aplicativos Android. A Seção III trata do processo de teste de aplicativos Android. Consiste em 16 perguntas, a maioria das quais são obrigatórias (7 perguntas de múltipla escolha, 3 perguntas subjetivas e 6 perguntas na caixa de seleção). Essas perguntas tratam de como o processo de teste de software é realizado pelos participantes, bem como o uso de ferramentas para automatizar o processo de teste e se eles estão ou não satisfeitos com as ferramentas existentes. Finalmente, a Seção IV refere-se às considerações finais. 87\% das perguntas eram obrigatórias. A seguir, descrevemos os objetivos de cada seção do questionário.


\begin{itemize}
    \item \textbf{Perfil dos entrevistados:} pretendemos identificar o perfil dos entrevistados, com informações sobre sexo, idade, escolaridade, área de formação, certificações de teste, nível de conhecimento e experiência em testes de software e, especificamente, em aplicativos Android.
    
    \item \textbf{Foco da atividade de teste:} procuramos identificar se os testes realizados são funcionais ou estruturais. Além disso, queremos identificar os tipos de testes realizados nos projetos atuais, como: teste de \ac{GUI}, teste de unidade, teste de integração, teste de regressão, etc.
    
    \item \textbf{Automação do processo de teste:} pretendemos identificar se os testes são realizados de forma manual ou automatizada. Além disso, queremos identificar o suporte ferramental utilizado pelos entrevistados. No questionário, listamos ferramentas para testar aplicativos Android, encontradas na literatura e aplicadas na indústria. Portanto, poderíamos investigar se as ferramentas listadas na literatura também são conhecidas e utilizadas pelos entrevistados. A Tabela \ref{table:androidtools} mostra o conjunto de ferramentas de teste para aplicativos Android sobre as quais perguntamos aos entrevistados.
    
    \item \textbf{Compreender o processo de teste de regressão:} pretendemos identificar se os nossos participantes realizam testes de regressão após realizar manutenção em aplicativos Android. Estamos interessados em descobrir se: (i) o processo é feito manualmente ou automatizado, (ii) é aplicada alguma técnica de seleção de caso de teste, (iii) são utilizadas ferramentas, encontradas na literatura ou aplicadas na indústria, (iv) se estão satisfeitos com as ferramentas existentes; e (v) se e por que consideram a execução de testes de regressão como relevante.
    
\end{itemize}

\subsection{Projeto Piloto}

Aplicamos um piloto com três participantes, dois profissionais e um pesquisador, todos trabalhando com testes de aplicativos Android. Para a versão piloto do questionário, adicionamos uma seção de avaliação em que os participantes puderam fazer contribuições sobre a estrutura do questionário e indicar sugestões de melhorias. A seção de avaliação apresentou questões relacionadas à adequação das questões em relação ao tema da pesquisa; o número de perguntas a serem respondidas; o tempo necessário para responder ao formulário; e a relevância da pesquisa, como segue:


\begin{enumerate}
    \item Em uma escala de 1 a 5, onde 1 é igual a ruim e 5 é igual a ótimo, como você avalia o número de perguntas no formulário?
    \item Em uma escala de 1 a 5, onde 1 é igual a ruim e 5 é igual a ótimo, como você avalia a adequação das perguntas do formulário ao tema "Processo de teste de aplicativos para Android"?
    \item Cite pontos para melhorias.
    \item Cite pontos positivos do questionário.
    \item Deixe um comentário sobre o questionário.
\end{enumerate}

O \textit{feedback} sugeriu pequenas modificações na estrutura do questionário. As mudanças incluíram: adicionar opções de resposta a várias perguntas, alterar palavras para melhorar o entendimento e destacar o objetivo da pergunta para diferenciar questões que têm contextos semelhantes. Com base nas melhorias identificadas, uma segunda versão do questionário foi gerada.


\subsection{Distribuição do Questionário}

Para criar o questionário, usamos a ferramenta online Formulários Google\footnote{\url{http://docs.google.com/forms}}. O processo de ampla divulgação desta pesquisa foi realizado em 20 de janeiro de 2020 e ficou disponível por 30 dias. O link do questionário foi enviado por e-mails individuais, listas de e-mail e mídias sociais. O questionário também foi enviado por e-mail para empresas que trabalham com o desenvolvimento de aplicativos Android. O formulário foi enviado juntamente com um texto de convite, com informações básicas sobre o objetivo do estudo e a importância do estudo. Os participantes também foram informados sobre as políticas de privacidade do estudo de maneira clara e detalhada.


\section{Resultados}\label{resultssurvey}

O \textit{Survey} coletou 32 respostas. Em relação à demografia dos entrevistados, 78,1\% eram homens e 21,9\% mulheres; 37,5\% estavam entre 20 e 25 anos, 28,1\% tinham entre 26 e 30 anos, 18,8\% tinham entre 31 e 35 anos, e 15,6\% tinham mais de 35 anos.

Em relação ao nível de escolaridade dos entrevistados, 18,8\% possuíam ensino médio; 3,1\% possuíam curso técnico; 15,6\% eram estudantes de graduação; 34,4\% possuíam diploma de ensino superior; 6,2\% tinham cursos de especialização; 18,8\%
eram Msc.; e 3,1\% eram doutores. Além disso, 87,5\% eram provenientes de Ciência da Computação ou áreas afins, 3,1\% de Engenharia Elétrica, 3,1\% de outras engenharia, e 6,3\% possuíam formação superior em outras áreas. A respeito de cursos e certificações em testes, 56,2\% possuem pelo menos um deles. A maioria dos cursos citados foram nas seguintes ferramentas: Angular, Apium, JUnit, Selenium, e SonarQube. A certificação citada foi a \textit{Certified Tester Foundation} (CTFL) da \textit{International Software Testing Qualifications Board}.

Em relação ao seu papel em projetos Android, 18,8\% dos entrevistados declararam que trabalham como autônomos; 62,5\% trabalha / pesquisa ou já trabalhou em uma empresa de software, 40,6\% eram estudantes da área e 3,1\% afirmaram que
eles nunca trabalharam em projetos de teste para aplicativos Android. Os entrevistados foram de sete estados diferentes no Brasil.

Em relação à experiência de trabalho em teste de software, 25\% dos entrevistados afirmaram ter até 1 ano de experiência, 28,1\% tinham entre 1 e 3 anos experiência, 28,1\% possuíam entre 3 e 5 anos de experiência, 15,6\% entre 5 e 10 anos e apenas 3,2\% tinham mais de 10 anos de experiência. Considerando sua experiência com testes de aplicativos Android, 56,3\% tinham até 1 ano de experiência, 21,9\% tinham entre 1 e 3 anos de experiência, 15,6\% entre 3 e 5 anos de experiência e 6,2\% entre 5 e 10 anos de experiência.

Quando perguntados sobre seu nível de conhecimento em testes de aplicativos Android, comparados para outros profissionais, 15,6\% responderam que consideravam seu nível de
muito baixo, 25\% baixo, 21,9\% regular, 21,9\% bom e 15,6\% excelente.

Além disso, em relação à principal função do projeto atual na empresa em que trabalham, foram mencionados: analista de qualidade, analista de sistemas, analista de teste ou engenheiro de teste, design e arquitetura de teste automatizado, coordenador de inspeção, coordenador de inspeção, desenvolvedor e gerente de projeto.

Em relação às atividades que mais realizam, 75\% trabalham com criação e design de casos de teste, 84,4\% trabalham com a execução de casos de teste, 3,1\% trabalham com automação de teste e 3,1\% trabalham com o relatório de teste para o cliente. Para esta pergunta, os entrevistados podem verificar mais de uma opção.

\subsection{Questões de Pesquisa}

Em seguida, relatamos os resultados da pesquisa, com base no conjunto das questões de pesquisa apresentadas anteriormente na na Seção \ref{rqsurvey}.

\begin{enumerate}[label=\bf QP\arabic*]
    
\item \textbf{Foco da atividade de teste:}
    
Observamos que a maioria dos testes realizados pelos entrevistados são testes funcionais, correspondendo a 90,6\%. Os testes estruturais são realizados por 25\% dos entrevistados. Os testes de “caixa cinza”, nos quais o acesso ao código fonte é parcial, são realizados por 21,9\% dos entrevistados.

Quando perguntado sobre os tipos de testes que eles executam nos projetos Android, o teste da GUI foi o mais executado e foi seguido pelos testes de validação, regressão, sistema, integração e unidade. A Figura \ref{figure:s_tipostestes} mostra a distribuição das categorias de teste. Vale ressaltar que na questão referida os respondentes puderam selecionar mais de uma opção.
    
\item \textbf{Automação do processo de teste:}

34,4\% dos entrevistados responderam que realizam testes do Android manualmente, sem usar nenhuma ferramenta de suporte; 15,6\% afirmaram realizar testes automatizados; e 50\% responderam que realizam testes manuais e automatizados. Além disso, além das ferramentas apresentadas no questionário (da literatura e da indústria), os entrevistados também podem indicar outros. A Figura \ref{figure:s_ferramentastestes} mostra que as ferramentas mais usadas são Appium, UI Automator, Monkey e Katalon. Além disso, alguns entrevistados relataram que é uma prática comum para suas empresas criar sua própria ferramenta de teste, para atender às suas demandas. Nesta pergunta, mais de uma opção pode ser verificada.
    
\item \textbf{Teste após manutenção (atualização) de apps Android:}

Quando perguntados sobre a execução de qualquer tipo de teste após as tarefas de manutenção, 40,6\% responderam que sempre realizam testes, 12,5\% responderam que realizam frequentemente, 21,9\% responderam que às vezes realizam testes, 9,4\% responderam que raramente testam e 15,6\% responderam que nunca testam.

Em relação à automação do processo de teste durante a manutenção, 25\% responderam que os testes são manuais, 15,6\% responderam que os testes são realizados de forma automatizada, 53,1\% responderam que realizam testes manualmente e automatizados e 6,3\% responderam que não testam os aplicativos após realizar a manutenção.

A literatura sobre técnicas de seleção para testes de regressão aponta alguns procedimentos que visam reduzir o número de testes a serem reexecutados [3] - [10]. Nesse sentido, pedimos aos participantes que entendessem quais desses procedimentos são realizados durante o processo de teste após o lançamento de uma nova versão do aplicativo.

Observamos que o processo mais executado é executar novamente todos os casos de teste da versão original para testar a versão atualizada do aplicativo. Essa é a técnica essencial do processo de teste de regressão, embora não seja a mais eficiente, principalmente se o número de casos de teste aumentar [26]. A Figura 3 mostra esses resultados.


Quanto ao uso de ferramentas no processo de teste de regressão, 31,2\% responderam que não usam ferramentas, enquanto 68,8\% responderam que utilizam ferramentas no processo de teste de regressão. Além disso, indicamos no questionário um conjunto de ferramentas disponíveis para testes de regressão de aplicativos Android e solicitamos aos participantes que indiquem quais eles costumam usar. Eles estavam livres para mencionar outras ferramentas, que não estão listadas no questionário. A Figura 4 apresenta o principal resultado para esta pergunta. Observamos que ATOM e Redroid são as ferramentas mais utilizadas (na literatura) e MonkeyRunner (na indústria).

Em relação à opinião dos participantes sobre a realização de testes após a manutenção, 96,9\% responderam que consideram isso uma tarefa relevante. Também perguntamos quais são os principais fatores para suas empresas não executarem testes após a manutenção dos aplicativos. Além dos fatores indicados na pesquisa, os entrevistados apontaram outros fatores, como “cultura no processo de desenvolvimento, acreditando que a fixação de uma área não afetará outra” e “pressão do tempo do cliente para lançar o produto no mercado”. mercado para não perder o tempo de colocação no mercado ”.

A Figura 5 mostra os principais fatores relatados pelos entrevistados. Observamos que o fator que mais influencia o teste de regressão por não ser realizado é o tempo reduzido para entregar uma versão atualizada do aplicativo ao cliente.

Quando perguntados se as ferramentas utilizadas atendem às necessidades de realizar testes de regressão, 9,4\% responderam que nunca se encontram, 9,4\% responderam que raramente se encontram, 12,5\% responderam que às vezes se encontram, 18,8\% responderam que geralmente se encontram, 12,5\% responderam que sempre se encontram e 37,4\% responderam que não testam o aplicativo após a manutenção. Quando perguntados se estavam satisfeitos com os recursos oferecidos pelas ferramentas de teste de aplicativos Android existentes, 43,8\% responderam que sim e 56,2 \% responderam que não.

\end{enumerate}

\section{Discussões}\label{discussions}



\textbf{Perfil dos entrevistados:} 
    
\textbf{Foco da atividade de teste:} 
    
\textbf{Automação do processo de teste:} 

\section{Ameaças à Validade}\label{threatstovalidity}



\section{Síntese do capítulo}

O presente capítulo apresentou o \textit{Survey} utilizado como instrumento de pesquisa para compreender qual a perspectiva de acadêmicos e profissionais sobre técnicas de seleção de teste de regressão implementadas por ferramentas para projetos de \ac{APPS} Android. Na Seção \ref{rqsurvey} definimos as questões de pesquisa. Na Seção \ref{rmsurvey} apresentamos a metodologia utilizada. A Seção \ref{resultssurvey} apresenta os resultados obtidos,e na Seção \ref{discussions} as devidas discussões. Por fim, na Seção \ref{threatstovalidit} apontamos possíveis ameaças à validade deste estudo.




























