\xchapter{\textit{Survey}}{}
\acresetall 

Após apresentar as técnicas de regressão de testes de regressão encontradas na literatura, e as ferramentas disponíveis para testes de \ac{APPS} Android, essas informações serão usadas neste \textit{survey} com o objetivo de compreender como o processo de testes de apps Android é realizado por estudante e profissionais que trabalham nessa área, bem como, verificar se as ferramentas encontradas tanto na literatura quanto no mercado são utilizadas por eles, como forma de responder a QP3 da presente investigação. Neste capítulo encontram-se detalhes sobre o planejamento, execução e resultados obtidos.

\section{Apresentação e Questões de Pesquisa}

Segundo \cite{PSK2001} um \textit{survey} é um sistema abrangente que tem como objetivo coletar informações para descrever, comparar ou explicar conhecimentos, atitudes e comportamentos. A literatura apresenta alguns trabalhos que tem trazem a aplicação de teste de regressão na indústria \cite{Ali2019}, \cite{630875}.


O presente \textit{survey} tem como objetivo principal compreender como profissionais / pesquisadores realizam o processo de teste dos \ac{APPS} desenvolvidos para Android, após realizarem uma manutenção / atualização. E, como objetivos específicos:

\begin{itemize}
    \item Avaliar se os testes são realizados de forma manual, automatizada ou ambas;
    \item Identificar quais as ferramentas utilizadas no processo de teste de \ac{APPS} Android;
    \item Verificar se as ferramentas propostas na literatura para testes de \ac{APPS} Android são utilizadas pelos participantes;
    \item Verificar se as ferramentas encontradas no mercado / literatura cinza são utilizadas pelos participantes;
    \item Verificar se as ferramentas utilizadas no processo de teste atendem as necessidades, no que refere-se a redução de tempo e esforço, garantindo a qualidade da atividade de teste e de teste de regressão de \ac{APPS} Android.
\end{itemize}

Neste estudo, será adotado como ferramenta de pesquisa, um questionário para coletar informações a cerca de como é realizado o processo de teste de regressão no desenvolvimento de \AC{APPS} Android.

\subsection{Questões de Pesquisa}
Para alcançar o objetivo proposto por este estudo, foi definida à seguinte questão de pesquisa principal:
\leavevspace 

\begin{center}
    \noindent\fbox{ 
        \parbox{.8\textwidth}
        {
        \begin{center}
            
            \textbf{Como profissionais / estudantes realizam o processo de teste de \ac{APPS} Android após realizarem uma manutenção / atualização?}
            
        \end{center}
        }
    }
\end{center}

\vspace{.5em}

Esta questão foi refinada nas questões específicas:
\vspace{.5em}

\begin{enumerate}[label=\bf QP\arabic*,leftmargin=1.8cm]
    
    \item \textbf{Qual a visão dos profissionais / estudantes no que refere-se a automatização de teste de \ac{APPS} Android?} Essa questão de pesquisa busca identificar se o processo de teste é feito de forma manual, automatizada ou de ambas as formas.
    
    \item \textbf{As ferramentas propostas na literatura para testes de \ac{APPS} Android são utilizadas pelos participantes?} Essa questão de pesquisa busca verificar se as ferramentas que a literatura propõe para realização de testes de \ac{APPS} Android são utilizadas pelos participantes.
    
    \item \textbf{As ferramentas encontradas no mercado e na literatura cinza para testes de \ac{APPS} Android são utilizadas pelos participantes?} Essa questão de pesquisa busca identificar se as ferramentas encontradas no mercado e na literatura para testes de \ac{APPS} Android são utilizadas pelos participantes.
    
    \item \textbf{Ao realizarem uma manutenção / atualização em um \ac{APPS}, realizam testes que garantam que o aplicativo não teve seu comportamento alterado?} Essa questão de pesquisa pretende identificar se a empresa realiza teste de regressão.
    
    \item \textbf{Qual(is) motivo(s) leva(m) a não realização do teste de regressão?} Essa questão de pesquisa busca identificar qual ou quais o(s) motivo(s) leva(m) os participantes a não realizarem teste de regressão.
    
    \item \textbf{As ferramentas para realização de testes em \ac{APPS} Android atendem as necessidades do mercado?} Essa questão de pesquisa visa identificar se as ferramentas que realizam teste de \ac{APPS} Android, disponíveis tanto na literatura, quanto no mercado, atendem as perspectivas e as demandas, na visão dos participantes. Ainda, busca identificar potencial de \textit{features} a serem implementadas nas ferramentas existentes e/ou em novas ferramentas.
    
\end{enumerate}


\section{Planejamento}

Para o planejamento deste \textit{survey} foi utilizado o design de observação proposto por \cite{Kitchenham:2002:PSR:566493.566495} e \cite{Kasunic2005DesigningAE} utilizado para identificar o comportamento dos profissionais tendo como base eventos já ocorridos. O presente \textit{survey} elencou ferramentas para automatização de testes de \ac{APPS} Android tanto utilizados no mercado, quanto indicados na literatura. O público alvo foram estudantes e profissionais que trabalham em projetos realizando testes de \ac{APPS} Android. 

O questionário foi dividido em quatro seções:
\begin{enumerate}
    \item A primeira refere-se ao termo de consentimento livre e esclarecido;
    \item A segunda destina-se a identificar o perfil dos respondentes, com 13 questões referentes a idade, gênero, escolaridade, conhecimento e experiência em teste e testes de \ac{APPS} Android.
    \item A terceira, com 16 questões, tem como objetivo identificar como é feito o processo de teste em \ac{APPS} Android. Busca-se verificar se é realizado teste de regressão, após manutenção (atualização) do app. Identificar se utilizam ferramentas no processo de teste. Levantar necessidades de implementação de novas ferramentas ou novas \textit{features} nas ferramentas existentes. O termo ATUALIZAÇÃO (MANUTENÇÃO) foi utilizado como referência ao teste de regressão.
    \item A quarta seção destina-se saber se o respondente tem interesse em receber os resultados.
\end{enumerate}

As perguntas do questionário estão disponíveis no apêndice \ref{sec:formulariopesquisa}. O símbolo \textbf{*}, encontrado no final da sentença, indica que a questão possui resposta obrigatória. As questões dividem-se em: questões de múltipla escolha, onde pode ser escolhido apenas uma alternativa; questões caixa de seleção, que pode ser marcado mais de uma alternativa; e questões subjetivas, que são questões abertas. As respostas obtidas foram tabuladas e estão disponíveis no apêndice \ref{sec:resultadospesquisa}. 

\section{Piloto}

Para auxiliar no entendimento do questionário, e verificar possíveis melhorias a serem feitas, foi aplicada uma versão piloto com três participantes, sendo dois com experiência profissional e um com experiência acadêmica, todos, na área de testes de \ac{APPS} Android. Na versão piloto do questionário, foi adicionada uma seção de avaliação, onde os participantes puderam dar contribuições referente a estrutura do questionário e indicar sugestões de melhorias. A partir destas respostas, foram realizadas modificações no questionário, com o objetivo de torná-lo mais compreensível para os participantes.


\section{Participantes}

Para criação do questionário foi utilizado a ferramenta online Google Forms\footnote{\url{http://docs.google.com/forms}}. Foi feita ampla divulgação desta pesquisa, sendo disponibilizado o link do questionário por e-mails individuais, lista de e-mails e mídias sociais, tais como whatsapp e linkedin. Foi ainda, encaminhado esse questionário para e-mails de empresas que trabalham com desenvolvimento de \ac{APPS} Android. O questionário ficou disponível para preenchimento durante o período de 20 de janeiro a 20 de fevereiro de 2020.


\section{Análise de Dados}


\section{Resultados}

\subsection{Informações Gerais}

\subsection{Processo de Teste}

\section{Discussão}

\section{Ameaças à validade}

\section{Síntese do capítulo}





