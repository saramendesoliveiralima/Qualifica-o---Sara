\xchapter{\textit{Survey}}{}
\acresetall 

\section{Apresentação e Questões de Pesquisa}

Segundo \cite{PSK2001} um \textit{survey} é um sistema abrangente que tem como objetivo coletar informações para descrever, comparar ou explicar conhecimentos, atitudes e comportamentos. A literatura apresenta alguns trabalhos que tem trazem a aplicação de teste de regressão na indústria \cite{Ali2019}, \cite{630875}.


O presente \textit{survey} tem como objetivo compreender como profissionais / pesquisadores realizam o processo de teste dos \ac{APPS} desenvolvidos para Android, após realizarem uma manutenção / atualização. E, como objetivos específicos:

\begin{itemize}
    \item Avaliar se os testes são realizados de forma manual, automatizada ou ambas;
    \item Identificar quais as ferramentas utilizadas no processo de teste de \ac{APPS} Android;
    \item Verificar se as ferramentas propostas na literatura para testes de \ac{APPS} Android são utilizadas pelos participantes;
    \item Verificar se as ferramentas encontradas no mercado / literatura cinza são utilizadas pelos participantes;
    \item Verificar se as ferramentas utilizadas no processo de teste atendem as necessidades, no que refere-se a redução de tempo e esforço, garantindo a qualidade da atividade de teste e de teste de regressão de \ac{APPS} Android.
\end{itemize}

Neste estudo, será adotado como ferramenta de pesquisa, um questionário para coletar informações a cerca de como é realizado o processo de teste de regressão no desenvolvimento de \AC{APPS} Android.

\subsection{Questões de Pesquisa}
Para alcançar o objetivo proposto por este estudo, foi definida à seguinte questão de pesquisa principal:
\leavevspace 

\begin{center}
    \noindent\fbox{ 
        \parbox{.8\textwidth}
        {
        \begin{center}
            
            \textbf{Como profissionais / estudantes realizam o processo de teste de \ac{APPS} Android após realizarem uma manutenção / atualização?}
            
        \end{center}
        }
    }
\end{center}

\vspace{.5em}

Esta questão foi refinada nas questões específicas:
\vspace{.5em}

\begin{enumerate}[label=\bf QP\arabic*,leftmargin=1.8cm]
    
    \item \textbf{Qual a visão dos profissionais / estudantes no que refere-se a automatização de teste de \ac{APPS} Android?} Essa questão de pesquisa busca identificar se o processo de teste é feito de forma manual, automatizada ou de ambas as formas.
    
    \item \textbf{As ferramentas propostas na literatura para testes de \ac{APPS} Android são utilizadas pelos participantes?} Essa questão de pesquisa busca verificar se as ferramentas que a literatura propõe para realização de testes de \ac{APPS} Android são utilizadas pelos participantes.
    
    \item \textbf{As ferramentas encontradas no mercado e na literatura cinza para testes de \ac{APPS} Android são utilizadas pelos participantes?} Essa questão de pesquisa busca identificar se as ferramentas encontradas no mercado e na literatura para testes de \ac{APPS} Android são utilizadas pelos participantes.
    
    \item \textbf{Ao realizarem uma manutenção / atualização em um \ac{APPS}, realizam testes que garantam que o aplicativo não teve seu comportamento alterado?} Essa questão de pesquisa pretende identificar se a empresa realiza teste de regressão.
    
    \item \textbf{Qual(is) motivo(s) leva(m) a não realização do teste de regressão?} Essa questão de pesquisa busca identificar qual ou quais o(s) motivo(s) leva(m) os participantes a não realizarem teste de regressão.
    
    \item \textbf{As ferramentas para realização de testes em \ac{APPS} Android atendem as necessidades do mercado?} Essa questão de pesquisa visa identificar se as ferramentas que realizam teste de \ac{APPS} Android, disponíveis tanto na literatura, quanto no mercado, atendem as perspectivas e as demandas, na visão dos participantes. Ainda, busca identificar potencial de \textit{features} a serem implementadas nas ferramentas existentes e/ou em novas ferramentas.
    
\end{enumerate}



\section{Planejamento}

Para o planejamento desse \textit{survey} foi utilizado o design de observação proposto por \cite{Kitchenham:2002:PSR:566493.566495} utilizado para identificar o comportamento dos profissionais tendo como base eventos já ocorridos.


O público alvo deste \textit{survey} são profissionais que trabalham na indústria com testes de \ac{APPS} Android.


Para divulgar essa pesquisa far-se-á uso da seguinte metodologia:

\begin{enumerate}
    \item Elaboração do questionário a ser aplicado, vide Apêndice \ref{sec:formulariopesquisa};
    \item Criação de um formulário on-line com uso do pacote de aplicativo Google Docs, para disponibilizar o questionário;
    \item Validação do questionário a ser aplicado, por meio de realização de um projeto piloto;
    \item Divulgação do formulário on-line por meio de e-mail para empresas que trabalham com desenvolvimento / testes de \ac{APPS} Android;
    \item Coleta, análise e síntese dos dados, obtidos pelo preenchimento do formulário on-line.
\end{enumerate}


O questionário foi dividido em quatro partes:

\begin{enumerate}
    \item O primeiro refere-se ao termo de consentimento livre e esclarecido;
    \item O segundo destina-se a identificar o perfil dos respondentes, sendo composto por dez perguntas;
    \item O terceiro refere-se ao padrão de seleção e realização de teste de regressão;
    \item O quarto destina-se saber se o respondente tem interesse em receber os resultados e, se necessário, participar de outras etapas da pesquisa.
\end{enumerate}

\section{Piloto}

\section{Participantes}

\section{Análise de Dados}


\section{Resultados}

\subsection{Informações Gerais}

\subsection{Processo de Teste}

\subsection{Processo de Teste de Regressão}

\section{Discussão}

\section{Ameaças à validade}

\section{Síntese do capítulo}





