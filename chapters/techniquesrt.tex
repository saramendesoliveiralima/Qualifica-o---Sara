\xchapter{Técnicas de Seleção de Teste de Regressão}{}

Este capítulo traz um breve resumo sobre as técnicas de seleção de teste de regressão existentes na literatura, categorizando-as de acordo com sua inclusividade, precisão, eficiência e generalidade.

\section{Apresentação}

A atividade de teste é um das formas de garantia da qualidade de software, visto que, esta tem como principal objetivo encontrar defeitos no funcionamento de uma aplicação \cite{HIRAMA2011}. Os testes são realizados na maioria das vezes durante o ciclo de desenvolvimento. Porém, o processo contínuo de manutenção do software, quer seja para realização de alguma correção perfectiva, corretiva, adaptativa ou preventiva \cite{DBLP:series/springer/Mens08}, exige que sempre que uma nova versão do software seja disponibilizada, executem-se testes que garantam que as novas implementações não causaram efeitos não-intencionais e se o sistema ou componentes ainda estão funcionando em conformidade com os requisitos especificados \cite{159342}. Essa técnica de testes é denominada de teste de regressão.

O teste de regressão constitui grande maioria dos esforços de teste no desenvolvimento de um software comercial, e é parte essencial de qualquer processo de desenvolvimento de um software com qualidade. Com a necessidade cada vez mais crescente de evolução dos softwares desenvolvidos, e em vista de garantir a qualidade dos mesmos, a necessidade de estudos de técnicas de teste de regressão é de grande relevância. Para minimizar os esforços com o teste de regressão, a automatização desse processo é de fundamental importância \cite{Ammann:2008:IST:1355340}.

A técnica ou estratégia de teste de regressão mais usual é a reexecução de todos os casos de teste da versão original para testar a versão atualizada, porém, esta técnica / estratégia pode tornar-se inviável por exigir uma quantidade muito grande de tempo para sua execução completa, ainda mais, quando pensamos em um processo de manutenção contínuo de um software \cite{Graves:2001:ESR:367008.367020}.

Em virtude da necessidade de técnicas \textbf{eficientes}, que determinam o subconjunto apropriado de casos de testes a serem reexecutados, e de técnicas \textbf{precisas} que omitem casos de testes que não são reveladores de modificação \cite{Ammann:2008:IST:1355340}, ao longo de décadas diversos estudos são realizados para prover essas técnicas \cite{WHITE1991}, \cite{Graves:2001:ESR:367008.367020}, \cite{630875}, \cite{536955}, \cite{ENGSTROM201014},\cite{ENGSTROM201014}, \cite{KAZMI2017}, \cite{ROMANO201862}.

Este trabalho tem como objeto de estudo as técnicas de seleção. Estas técnicas podem ser categorizadas, para que sejam avaliadas e comparadas, em quatro categorias: inclusividade, precisão, eficiência e generalidade. A \textbf{inclusividade} mede o quanto uma técnica escolhe testes que farão com que o programa modificado produza uma saída diferente do programa original expondo falhas causadas por modificações. A \textbf{precisão} mede a capacidade de uma técnica evitar a escolha de testes que não farão com que o programa modificado produza resultados diferentes do programa original. A \textbf{eficiência} mede o custo computacional e, portanto, a praticidade de uma técnica. A \textbf{generalidade} mede a capacidade de uma técnica lidar com construções de linguagem realistas e diversificadas, modificações de código arbitrariamente complexas e aplicativos de teste realistas \cite{536955}.

\section{Conceitos Fundamentais}

As técnicas de seleção reduzem o custo do teste de um programa modificado, pois, reutilizam testes existentes e identificam partes do programa modificado ou sua especificação que deve ser testada. Estas técnicas diferem da técnica de reexecução de todos os casos de teste da versão original. Segundo \citeonoline{WHITE1991} uma técnica de seleção é mais econômica que a técnica de reexecução de todos os casos de teste, se o custo de selecionar um subconjunto reduzido de testes a executar for menor que o custo de executar os testes que a técnica de seleção omitir.

Os procedimentos fundamentais executados por uma técnica de seleção de teste de regressão são: \cite{536955}

\begin{enumerate}
    \item Selecionar $T'$ $\subseteq$ $T$, um conjunto de casos de testes para executar em $P'$.
    \item Testar $P'$ com $T'$, para estabelecer a correção de $P'$ em relação a $T'$.
    \item Se necessário, criar $T''$, um novo conjunto de casos de teste funcional ou estrutural para testar $P'$.
    \item Testar $P'$ com $T''$, para estabelecer a correção de $P'$ em relação a $T''$.
    \item Crie $T'''$, um novo conjunto de casos de testes e histórico de testes para $P'$, de $T$, $T'$, e $T''$.
\end{enumerate}

Executando essas etapas, uma técnica de seleção de teste de regressão aborda quatro problemas. A etapa 1 refere-se ao problema de selecionar um subconjunto de testes $T'$ de $T$ com o qual possa testar $P'$. A etapa 3 refere-se ao problema de cobertura, ou seja, identificar a necessidade de criação de testes adicionais. As etapas 2 e 4 tratam o problema de executar testes com eficiência, verificando se os resultados estão corretos. A etapa 5 refere-se ao problema de manutenção do conjunto de testes, atualizando e armazenando as informações de teste.

As técnicas de seleção de teste de regressão tem como foco o problema 1, ou seja, selecionar um subconjunto de teste de $T$, $T'$ com o qual testar $P'$, garantindo assim que as mudanças realizadas não modificaram o funcionamento do software. Essas técnicas localizam testes em $T$ que expõem falhas em $P'$.

Algumas premissas fundamentais em relação a técnicas de seleção de teste de regressão: \cite{536955}
\begin{itemize}
    \item Um teste $t$ detecta uma falha em $P'$ se ele faz com que $P'$ falhe. Neste caso pode-se dizer que $t$ é um revelador de falhas de $P'$.
    \item Um programa $P$ falha em $t$ se, quando $P$ for testado com $t$, $P$ produzir uma saída incorreta de acordo com $S$.
    \item Uma técnica de seleção de teste de regressão pode selecionar um subconjunto do conjunto de testes em $T$ que revelam falhas para $P'$. Nessas condições pode-se dizer que essa técnica não omite testes em $T$ que possam revelar falhas em $P'$.
    \item Um teste $t$ é revelador de modificações para $P$ e $P'$ se, e somente se, faz com que as saídas de $P$ e $P'$ sejam diferentes.
    \item O teste $t$ é considerado obsoleto para o programa $P'$ se e somente se $t$ especificar uma entrada para $P'$ que, de acordo com $S'$, é inválida para $P'$ ou $t$ especificar uma relação de entrada-saída inválida para $P'$.
\end{itemize}

\section{Técnicas de Seleção}

A literatura apresenta diversos estudos sobre técnicas de seleção de teste de regressão, tais como, revisões sistemáticas, \textit{survey}, estudos empíricos: \cite{WHITE1991}, \cite{536955}, \cite{630875}, \cite{Rosenblum97acomparative}, \cite{Rothermel2000},  \cite{Graves:2001:ESR:367008.367020},    \cite{ENGSTROM201014}, \cite{KAZMI2017}, \cite{ROMANO201862}.

\section{Síntese do Capítulo}
