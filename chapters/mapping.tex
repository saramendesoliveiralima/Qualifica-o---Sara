\xchapter{Mapeamento Sistemático}{}

\acresetall 


\section{Apresentação}\label{sec:apresentacaoret}

A literatura apresenta diversos estudos relacionados a técnicas de teste de regressão \textcolor{blue}{\cite{536955},
\cite{65194},
\cite{630875},
\cite{Graves:2001:ESR:367008.367020},
\cite{962562},
\cite{988497},
\cite{5954416},
\cite{6339502},
\cite{Yoo:2012:RTM:2284811.2284813},
\cite{6569773},
\cite{ENGSTROM201014},
\cite{7427895},
\cite{7832883},
\cite{Do2016RedroidAR},
\cite{7927972},
\cite{8377661},
\cite{8424973},
\cite{ROMANO201862},
\cite{Choi:2018:DMA:3180155.3180173}}. Em virtude do crescimento do mercado de desenvolvimento para Android, \textcolor{blue}{este mapeamento sistemático tem como objetivo realizar uma análise das técnicas de teste de regressão para \ac{APPS} Android.} 

\textcolor{blue}{Este mapeamento sistemático segue a metodologia proposta por} \cite{Kitchenham:2015:ESE:2994449} e será apresentado em três etapas:

\begin{itemize}
    \item Planejamento
    \item Execução
    \item Análise e Síntese dos Dados
\end{itemize}

\section{Planejamento}\label{sec:planejamentoret}

\textcolor{blue}{Na etapa de planejamento, foi definido um protocolo para identificação e seleção dos estudos candidatos, que atendam os critérios estabelecidos para realização deste.}

\subsection{Questões de Pesquisa}

Este mapeamento sistemático tem como objetivo conhecer o estado-da-arte sobre \textcolor{blue}{técnicas de teste de regressão para Android}, propondo respostas para um conjunto de perguntas. As questões de pesquisa buscam encontrar procedimentos eficientes para técnicas de teste de regressão na prática para Android. \textcolor{blue}{Procuramos por técnicas de teste de regressão para Android de estudos avaliados empiricamente e que sejam implementadas por ferramentas.}

As perguntas são:

\begin{enumerate}[label=\bf QP\arabic*,leftmargin=2.1cm]
  \item \textbf{Quais técnicas de teste de regressão para Android foram avaliadas empiricamente?} A presente questão de pesquisa tem como objetivo levantar as \textcolor{blue}{técnicas de teste de regressão} para Android encontradas na literatura com avaliação empírica.
  
  \item \textbf{Quais técnicas de teste de regressão para Android são implementadas por ferramentas?} A presente questão de pesquisa tem como objetivo levantar as técnicas de teste de regressão para Android encontradas na literatura que foram implementadas por ferramentas.
  
  \item \textbf{Essas técnicas podem ser classificadas e, em caso afirmativo, como?} A presente questão de pesquisa tem como objetivo identificar se as técnicas encontradas podem ser agrupadas/classificadas de acordo com algum critério.

  \item \textbf{Os estudos trazem ferramental de apoio para realização de Teste de Regressão?} A presente questão de pesquisa tem como objetiva levantar possíveis ferramentas para realização de Teste de Regressão utilizadas pelas técnicas levantadas na literatura.
  
  \item \textcolor{blue}{\textbf{Os estudos especificam qual tipo de teste é realizado (unidade, integração, validação, sistemas, interface)?} A presente questão de pesquisa tem como objetivo identificar quais tipos de testes são realizados.}
\end{enumerate}

As respostas a essas questões de pesquisa são pesquisadas na literatura publicada usando os procedimentos de revisões sistemáticas da literatura, como proposto por \cite{Kitchenham07guidelinesfor} conforme Tabela \ref{table:PICO}.

\begin{table}[h!]
    \footnotesize
    \centering
    \caption{Critérios PICOC}
    \label{table:PICO}
    \def \arraystretch{1.3}
    \begin{tabular}{m{2.3cm}m{9cm}}
        \toprule
        \textbf{População} & Pesquisadores, desenvolvedores e testadores de software\\
        \textbf{Intervenção} & Técnicas de Teste de Regressão\\ 
        \textbf{Comparação} & \textit{Não aplicável}\\
        \textbf{Saídas} & Técnicas de Teste de Regressão para Android\\
        \textbf{Contexto} & Avaliadas Empiricamente / \textcolor{blue}{Implementadas por Ferramentas}\\
        \bottomrule
    \end{tabular}
\end{table}

\subsection{Estratégias de Busca}

Após adquirir conhecimento fundamental na área de pesquisa com base em revisões bibliográficas da literatura realizadas previamente, definimos uma estratégia de busca dividindo-a em três etapas:


\begin{enumerate}
    \item \textbf{Busca Automática}: Busca automática realizada nas bases de dados definidas na Tabela \ref{table:DATABASES}.
    
\begin{table}[h!]
    \centering
    \caption{Bases de Dados}
    \label{table:DATABASES}
    \footnotesize
    \def \arraystretch{1.2}
    \begin{tabular}{ll}
    \toprule
    \bf Biblioteca Digital & \bf URL\\
    \midrule
    \textit{ACM Digital Library} & https://dl.acm.org/\\
    \textit{DBLP} & https://dblp.uni-trier.de/\\
    \textit{IEEE Xplore} & https://ieeexplore.ieee.org/Xplore/home.jsp\\ 
    \textit{Scopus} & https://www.scopus.com/home.uri\\
    \textit{Science Direct} & https://www.sciencedirect.com/\\
    \bottomrule
    \end{tabular}
\end{table}


\begin{table}[h!]
    \centering
    \caption{String de Busca}
    \label{table:STRINGBUSCA}
    \footnotesize
    \def \arraystretch{1.2}
    \begin{tabular}{m{12cm}}
    \toprule
    \centering
    \textcolor{blue}{\textit{
    (``regression test'' OR ``regression testing'' OR ``test'' OR ``testing'') AND (``Android apllications'' OR ``Android APPS'' OR ``Android'')\\}}
    \bottomrule
    \end{tabular}
\end{table}

    \item \textbf{Busca Manual}: Busca manual realizada nas principais conferências da áreas de Engenharia de Software e de Teste.
 
    \item \textit{\textbf{Snowballing}}: Análise das referências dos artigos identificados na busca automática e na busca manual.

\end{enumerate}


\subsection{Critérios para Seleção de Estudos Primários}

Para inclusão ou exclusão dos estudos serão utilizados os critérios abaixo:

\begin{itemize}
    \item \textbf{Critérios de Inclusão}: 
    \begin{itemize}
        \item Estudos na área de Computação e Engenharia de Software
        \item Estudos que tratem sobre Técnicas de Teste de Regressão para Android
        \item Estudos que possuam avaliação empírica
        \item Estudos que apresentam técnicas implementadas por ferramentas
        \item Estudos escritos na língua inglesa
    \end{itemize} 

    \item \textbf{Critérios de Exclusão}: 
    \begin{itemize}
        \item Estudos duplicados
        \item Estudos sem comprovação empírica
        \item Estudos que apresentam técnicas que não são implementadas por ferramentas
        \item Estudos não escritos em inglês
        \item Versão reduzida de outros estudos
    \end{itemize}
\end{itemize}


Para que um estudo seja aceito, precisa contemplar todos os critérios de inclusão. Um estudo será excluído se atender pelo menos um critério de exclusão. O processo de seleção dos artigos será feita através do processo a seguir:

\begin{itemize}
    \item Excluir estudos candidatos por título ou resumo
    \item Excluir estudos candidatos depois de ler alguma seção
    \item Excluir estudos candidatos depois de ler todo o artigo
\end{itemize}


\subsection{Estratégia de Extração de Dados}

Após etapa de seleção de estudos, para cada trabalho restante, deverá ser extraído os dados relacionados na Tabela \ref{table:EXTRACAO}.


% https://www.sciencedirect.com/science/article/pii/S0950584914000834?via%3Dihub

\subsection{Síntese dos Dados Extraídos}

O processo de síntese de dados deverá responder as questões de pesquisa propostas neste mapeamento sistemático. O método utilizado será a síntese narrativa.


\subsection{Resultados Esperados}

Ao final deste mapeamento sistemático, pretende-se alcançar os seguintes resultados:

\begin{itemize}
    \item Levantar as técnicas de teste de regressão para Android, com comprovação empírica e que sejam implementadas por ferramentas.
    
    \item Classificar quais tipos de testes são abordados nos estudos.
    
    \item \textcolor{blue}{Identificar o ferramental de apoio utilizado na realização de teste de regressão em \ac{APPS} Android.}
\end{itemize}

\section{Resultados Preliminares}\label{sec:resultadospreliminares}

A atividade executada até o presente momento foi:

\begin{enumerate}
    
    \item Elaboração do protocolo da revisão estruturada de literatura
    
\end{enumerate}

\section{Próximos Passos}\label{sec:resultadosesperados}

As próximas etapas a serem executadas serão:

\begin{enumerate}

    \item Busca por estudos candidatos por meio da aplicação das strings de busca, conforme definido na Tabela \ref{table:STRINGBUSCA} nas bases de dados conforme definido na Tabela \ref{table:DATABASES}.
    
    \item Busca por estudos candidatos em conferências das áreas de engenharia de software e teste.
    
    \item Busca por estudos candidatos através do método do \textit{Snowballing}.

    \item Avaliação dos estudos candidatos, aplicando os critérios de exclusão, após leitura dos estudos por: título ou resumo; ou, depois de ler alguma seção.

    \item Realizar a extração dos dados dos estudos selecionados;
   
    \item Realizar a análise e síntese dos dados dos estudos selecionados.
\end{enumerate}




