\appendix
\xchapter{Questionário \textit{survey}}{}\label{sec:formulariopesquisa}
%sem preambulo

Prezado colaborador, este questionário destina-se a coletar dados para um trabalho acadêmico de mestrado em Ciência da Computação da Universidade Federal da Bahia (UFBA), com objetivo de investigar a aplicação de teste de regressão no desenvolvimento de \ac{APPS} Android. A sua participação é de fundamental importância.


\begin{enumerate}[label=\bf A\arabic*,leftmargin=1.8cm]

    \item \textbf{TERMO DE CONSENTIMENTO LIVRE E ESCLARECIDO}
    


    \begin{enumerate}[label= \arabic*]
        \item O/A senhor(a) está sendo convidada(o) a participar voluntariamente da pesquisa sobre aplicação de teste de regressão no desenvolvimento de \ac{APPS} Android.
        \item Sua participação não é obrigatória.
        \item A qualquer momento você pode desistir de participar e retirar seu consentimento.
        \item Sua recusa não trará nenhum prejuízo em sua relação com a pesquisadora ou com a instituição.
        \item Este formulário tem por objetivo investigar a aplicação de teste de regressão no desenvolvimento de \ac{APPS} Android.
        \item Sua participação neste formulário consistirá em responder questões objetivas e subjetivas.
        \item Sua identificação é opcional, ou seja, você não precisa informar nome ou e-mail caso assim deseje.
        \item A aplicação do formulário está sendo realizada por Sara Mendes Oliveira Lima, estudante da pós graduação em Ciência da Computação na Universidade Federal da Bahia, sob a supervisão do Prof. Dr. Ivan C. Machado.
        \item Os benefícios relacionados à sua participação estão apenas em contribuir com a pesquisa científica. Será permitido acesso aos resultados desta pesquisa por meio da dissertação ou publicações científicas realizadas a partir desse estudo.
        \item As informações pessoais obtidas através desta pesquisa serão confidenciais e não serão distribuídas ou divulgadas pela pesquisadora.
        \item Os dados coletados neste formulário não serão divulgados de forma a possibilitar sua identificação.
        \item Ao continuar respondendo este questionário, o/a senhor(a) concorda com as informações aqui descritas, porém a qualquer momento o/a senhor(a) poderá interromper a pesquisa sem ônus algum.
        \item Por gentileza, responda este formulário apenas se o/a senhor(a) trabalha com testes de \ac{APPS} Android.
        \item Este questionário utiliza o pacote de aplicativo Google Docs, portanto a coleta e o uso de informações do Google estão sujeitos à Política de privacidade do Google (https://www.google.co.uk/policies/privacy/).
        \item Abaixo seguem os dados de contato dos responsáveis por esta pesquisa, com os quais você pode tirar suas dúvidas sobre sua participação. Pesquisadores responsáveis: Sara Mendes Oliveira Lima - lima.sara@ufba.br, Ivan C. Machado, Ph.D. - ivan.machado@ufba.br (supervisor). Universidade Federal da Bahia (UFBA) - Instituto de Matemática e Estatística - Departamento de Ciência da Computação - Av. Adhemar de Barros, s/n, sala 280, Ondina, 40170-110, Salvador – BA
    \end{enumerate}
    
    
    Salvador - Ba, Janeiro de 2020.\\
    
    
    
    \textbf{Declaro que entendi os objetivos, riscos e benefícios de minha participação na pesquisa.*}

    \begin{itemize}
        \item \textbf{Concordo}
    \end{itemize}
    
    

    \item \textbf{PERFIL DO PARTICIPANTE}
    
    Essa sessão tem como objetivo identificar o perfil do respondente, tipo de trabalho realizado e aptidões técnicas.
    
     \begin{enumerate}[label= \arabic*]
     
        \item \textbf{Identidade de gênero:}
        \begin{itemize}
            \item Homem
            \item Mulher
            \item Outros
        \end{itemize}
        
        \item \textbf{Idade:}\\
        (questão subjetiva)
        
        \item \textbf{Titulação:}
        \begin{itemize}
            \item Nível médio completo
            \item Curso Técnico completo
            \item Graduação completa
            \item Especialização completa
            \item Mestrado completo
            \item Doutorado completo
            \item Pós-Doutorado
            \item Outros
        \end{itemize}
        
        \item \textbf{Área de Formação Acadêmica:}
        \begin{itemize}
            \item Informática, Computação ou áreas afins
            \item Engenharia Elétrica
            \item Engenharia de Produção
            \item Outras Engenharias
            \item Outros
        \end{itemize}
        
        \item \textbf{Você já fez algum curso / certificação na área de testes?}
        \begin{itemize}
            \item Sim
            \item Não
        \end{itemize}
        
        \item \textbf{Caso a pergunta acima tenha sido "Sim", quais foram esses cursos / certificações?}\\
        (questão subjetiva)
        
        \item \textbf{Em relação a projetos de \ac{APPS} Android, você:}
        \begin{itemize}
            \item Trabalha de forma autônoma
            \item Trabalha / pesquisa em uma empresa
            \item Estudante da área
            \item Outros
        \end{itemize}

        \item \textbf{Estado onde trabalha / estuda:}
        \begin{itemize}
            \item Acre
            \item Alagoas
            \item Amapá
            \item Amazonas
            \item Bahia
            \item Ceará
            \item Distrito Federal
            \item Espírito Santo
            \item Goiás
            \item Maranhão
            \item Mato Grosso
            \item Mato Grosso do Sul
            \item Minas Gerais
            \item Pará
            \item Paraíba
            \item Paraná
            \item Pernambuco
            \item Piauí
            \item Rio de Janeiro
            \item Rio Grande do Norte
            \item Rio Grande do Sul
            \item Rondônia
            \item Roraima
            \item Santa Catarina
            \item São Paulo
            \item Sergipe
            \item Tocantins
            \item Outros
        \end{itemize}
        
        
        \item \textbf{Experiência profissional na área de TESTES DE SOFTWARE:}
        \begin{itemize}
            \item Até 1 ano
            \item De 1 a 3 anos
            \item De 3 a 5 anos
            \item De 5 a 10 anos
            \item Acima de 10 anos
        \end{itemize}
        
        \item \textbf{Experiência profissional na área específica de TESTES DE APPS ANDROID:}
        \begin{itemize}
            \item Até 1 ano
            \item De 1 a 3 anos
            \item De 3 a 5 anos
            \item De 5 a 10 anos
            \item Acima de 10 anos
        \end{itemize}
        
        \item \textbf{Quando comparado a outros profissionais, como você considera o seu nível de conhecimento em TESTES DE \ac{APPS} ANDROID?}
        \begin{itemize}
            \item Muito baixo
            \item Baixo
            \item Regular
            \item Bom
            \item Excelente
        \end{itemize}
        
        \item \textbf{Caso seja funcionário de uma empresa, qual a sua principal função no projeto atual?}\\
        (questão subjetiva)
        
        
        \item \textbf{Qual(is) atividade(s) realiza no processo de teste de software:}
        \begin{itemize}
            \item Trabalha com criação / design de casos de teste
            \item Trabalha com execução de casos de teste
            \item Outro
        \end{itemize}        
        
     \end{enumerate}
     
     
    
     \item \textbf{PROCESSO DE TESTES DE \AC{APPS} ANDROID:}
     
     
     As perguntas a seguir referem-se a realização de teste de regressão. Utilize como base o comportamento encontrado no seu dia-a-dia de trabalho.
     
     \begin{enumerate}[label= \arabic*]
     
     \item \textbf{Tipo de técnica de teste executada:}
     \begin{itemize}
         \item Teste de caixa preta, também conhecido como teste funcional (sem acesso ao código fonte)
         \item Teste de caixa branca, também conhecido como teste estrutural (teste com acesso ao código fonte)
         \item Ambos tipos de teste
     \end{itemize}
     
    \item \textbf{Os testes são realizados de forma:}
     \begin{itemize}
        \item Manual
        \item Automatizada
        \item Ambas as formas
     \end{itemize}
    
    \item \textbf{Qual(is) tipos de testes a empresa realiza durante o processo de desenvolvimento \ac{APPS} Android?}
    \begin{itemize}
        \item Teste de unidade
        \item Teste de integração
        \item Teste de validação
        \item Teste de sistema
        \item Teste de recuperação
        \item Teste de proteção
        \item Teste de estresse
        \item Teste de desempenho
        \item Teste de conectividade
        \item Teste de segurança
        \item Teste em condições naturais
        \item Teste de certificação
    \end{itemize}
     
     \item \textbf{Quando é realizada uma manutenção (atualização) quer seja perfectiva, corretiva, adaptativa ou preventiva no APP Android, a empresa realiza algum tipo de teste com o objetivo de garantir que as mudanças realizadas não alteraram o comportamento funcional do APP?}
     \begin{itemize}
         \item Sim
         \item Não
     \end{itemize}
     
    \item \textbf{Caso realize o processo de teste, reexecuta todos os casos de teste da versão anterior ou seleciona um subconjunto de casos de teste?}
     \begin{itemize}
         \item Reexecuta todos os casos de teste
         \item Seleciona um subconjunto de casos de teste
     \end{itemize}
     
     \item \textbf{O processo de teste quando há uma atualização do APP é feita de forma:}
     \begin{itemize}
         \item Manual
         \item Automatizada
         \item Ambas as formas
     \end{itemize}
     
    \item \textbf{Caso seja feito de forma automatizada, utiliza alguma ferramenta?}
    \begin{itemize}
        \item Sim
        \item Não
    \end{itemize}
    
    \item \textbf{Caso utilize, qual é a ferramenta?}\\
    (questão subjetiva)
    
    \item \textbf{Qual a relevância de realizar testes ao atualizar uma versão do \ac{APPS} Android?}
    \begin{itemize}
        \item Irrelevante
        \item Pouco relevante
        \item Média relevância
        \item Relevante
        \item Muito relevante
    \end{itemize}
  
    \item \textbf{Como avalia o custo de realizar testes ao atualizar uma versão do \ac{APPS} Android?}
    \begin{itemize}
        \item Custo alto
        \item Custo médio
        \item Custo baixo
    \end{itemize} 

    \item \textbf{Qual o grau dos benefícios de realizar testes ao atualizar uma versão do \ac{APPS} Android?}
    \begin{itemize}
        \item Benefício alto
        \item Benefício médio
        \item Benefício baixo
    \end{itemize}

    \item \textbf{As ferramentas existentes no mercado de testes para \ac{APPS} Android atendem as necessidades?}
    \begin{itemize}
        \item Sim
        \item Não
        \item Parcialmente
    \end{itemize}   

    \item \textbf{Existe a necessidade de criação de novas ferramentas de testes para \ac{APPS} Android?}
    \begin{itemize}
        \item Sim
        \item Não
    \end{itemize}
    
    \item \textbf{Acredita que haja uma necessidade de pesquisas acadêmicas na área de testes para \ac{APPS} Android que forneçam soluções eficientes para a indústria?}
    \begin{itemize}
        \item Sim
        \item Não
    \end{itemize}
     
    \item \textbf{Acredita que a relação entre as pesquisas acadêmicas e a indústria é importante para desenvolver soluções eficientes que atendam as demandas do mercado de testes em \ac{APPS} Android?}
    \begin{itemize}
        \item Sim
        \item Não
    \end{itemize}
    
    \end{enumerate}
    
    
    \item \textbf{FINALIZAÇÃO:}
    
    \begin{enumerate}[label= \arabic*]
     
     \item \textbf{Gostaria de receber os resultados e, se necessário, participar de outras etapas dessa pesquisa?}
     \begin{itemize}
        \item Sim
        \item Não
    \end {itemize}
    
    \item \textbf{Caso tenha optado por sim na questão anterior, favor informar o e-mail para contato:}\\
    (questão subjetiva)
     
    \end{enumerate}
    
\end{enumerate}


